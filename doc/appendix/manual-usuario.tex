
\chapter{Manual de usuario e instalación}
\label{cha:manual-usuario-instalacion}

\section{Introducción}
\label{sec:intro-manual-de-usuario}

En este apéndice se va a dividir en dos secciones donde se va a explicar como instalar y configurar todas las herramientas necesarias para la puesta a punto del proyecto. Y por otro lado, un manual de usuario donde se va a explicar como ejecutar cada una de las funciones \pendiente{Pendiente extender esta parte}\textcolor{red}{blablabla \ldots}.
Cabe destacar que, como se indica en el apartado \ref{sec:intro-pliego}, el equipo donde se instalará todo el software para programar los distintos algoritmos debe de estar dotado con un sistema operativo Ubuntu 18.04.4 LTS.


\section{Guía de instalacion}
\label{sec:sec-guia-instalacion}

\subsection{Instalación de Git}
\label{subsec:instalacion-git}

\begin{lstlisting}
$ sudo apt update
$ sudo apt install git
\end{lstlisting}

Explicación rápida de que se git y como instalarlo con comandos

Explicar como instalar anaconda con comandos para Ubuntu

\subsection{Descarga del repositorio del proyecto}
\label{subsec:descarga-repo}

\begin{lstlisting}
$ git clone https://github.com/jmudy/yolov4-deepsort.git
\end{lstlisting}

\subsection{Instalación de Anaconda}
\label{subsec:instalacion-anaconda}

Explicar como instalar anaconda con comandos para Ubuntu

\subsection{Crear entorno virtual con Anaconda}
\label{subsec:creacion-entorno}

Aquí se explica como crear un entorno virtual de anaconda a partir de un .yml donde instalas todo de golpe ademas de CUDA y cuDNN

\section{Manual de usuario}
\label{sec:manual-usuario}

Aquí se especifica los comandos que se tienen que ejecutar para poner en funcionamiento el algoritmo de detección de objetos abandonados.


