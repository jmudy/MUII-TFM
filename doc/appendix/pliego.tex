\chapter{Pliego de condiciones}
\label{cha:pliego-de-condiciones}

\section{Introducción}
\label{sec:intro-pliego}

En este anexo se especifican el \textit{software} y \textit{hardware} utilizado en el desarrollo del proyecto. Por un lado tenemos el equipo A, que se ha empleado únicamente para la redacción del presente documento con \LaTeX. Por otro lado, la utilización del equipo B, cuya finalidad ha sido poder programar los algoritmos con mejor capacidad de computación dadas sus especificaciones.

\section{Características del equipo A}
\label{sec:caracteristicas-equipoa}

A continuación se detalla las especificaciones de \textit{software} y \textit{hardware} que tiene el primer equipo. Cabe destacar que, las especificaciones necesarias pueden ser menores a las indicadas ya que únicamente se va a utilizar para redactar el presente documento sobre un sistema operativo Windows 10.

\subsection{Especificaciones hardware del equipo A}
\label{subsec:especificaciones-hardware-equipoa}
\begin{itemize}
    \item Procesador: Intel\textregistered\ Core\textsuperscript{TM} i5-3570K @ 3.4Ghz Box
    \item Tarjeta gráfica: NVIDIA\textregistered\ Gigabyte GeForce GTX\textsuperscript{TM} 770 OC 2GB GDDR5
    \item Memoria intalada (RAM): G.Skill Ares DDR3 1600 PC3-12800 8GB 2x4GB CL9
    \item Almacenamiento 1: Samsung 860 EVO Basic SSD 500GB SATA3
    \item Almacenamiento 2: WD Blue 1TB SATA3
\end{itemize}

\subsection{Especificaciones software del equipo A}
\label{subsec:especificaciones-software-equipoa}
\begin{itemize}
    \item Utilización de un sistema operativo de 64 bits, Windows 10 Pro (compilación de SO 19042.804)
    \item Lenguaje de programación R 3.6.3 y recomendable utilizar RStudio 1.3.1093
    \item Procesador de textos \LaTeX\ con TexMaker 5.0.4 y MiKTeX-TeX 4.0 (MiKTeX 20.12)
    \item Software de control de versiones Git version 2.29.2.windows.1
\end{itemize}

\section{Características del equipo B}
\label{sec:caracteristicas-segun-equipob}

Para la correcta evaluación de los algoritmos programados se recomienda que las especificaciones del equipo sean igual o mayores a las detalladas. Tal y como se indica en la subsección \ref{subsec:creacion-entorno} del Apéndice \ref{cha:manual-usuario-instalacion}, es obligatorio disponer de una tarjeta NVIDIA con una capacidad de cálculo mayor a 3.5 para poder utilizar \gls{cuda}.

\subsection{Especificaciones hardware del equipo B}
\label{subsec:especificaciones-hardware-equipob}

\begin{itemize}
    \item Procesador: Intel\textregistered\ Core\textsuperscript{TM} i7-6700HQ @ 2.60 GHz
    \item Tarjeta gráfica: NVIDIA\textregistered\ GeForce GTX\textsuperscript{TM} 960M 2GB GDDR5
    \item Memoria intalada (RAM): 20GB
    \item Almacenamiento 1: Disco duro SSD 250GB
    \item Almacenamiento 2: Disco duro HDD 1TB
\end{itemize}

\subsection{Especificaciones software del equipo B}
\label{subsec:especificaciones-software-equipob}
\begin{itemize}
    \item Utilización de un sistema operativo de 64 bits, Ubuntu 18.04.4 LTS
    \item Entorno de trabajo Anaconda 4.9.1
    \item NVIDIA\textregistered\ \gls{cuda}\textsuperscript{TM} Toolkit 10.1.243 
    \item \gls{cudnn} v7.6.5 (November 5th, 2019), for \gls{cuda}\textsuperscript{TM} 10.1
    \item Lenguaje de programación Python\textsuperscript{TM} 3.7.0 y recomendable utilizar Visual Studio Code como editor de código fuente
\end{itemize}

Cabe resaltar que la utilización de dos equipos ha sido por simple comodidad. Para el desarrollo del proyecto se puede emplear un único equipo que tenga las mismas prestaciones o superiores al equipo B. Se ha empleado un equipo con Windows 10 porque ha facilitado el manejo de datos con Microsoft Office 365.