
\chapter{Presupuesto}
\label{cha:presupuesto}

\section{Introducción}
\label{sec:intro-presupuesto}

En este apéndice se muestra el presupuesto del proyecto. Para ello, en los siguientes apartados se detallarán el perfil de personal necesario para desarrollarlo, la duración estimada de cada una de las etapas que lo compone, o los recursos \textit{hardware} y \textit{software} utilizados, el coste de la mano de obra y finalmente el coste total.

\section{Equipo de trabajo}
\label{sec:equipo-presupuesto}

Para la realización del proyecto se necesita tener a un Ingeniero de Telecomunicaciones, Informático o Industrial especializado en percepción que esté altamente familiarizado con el uso de redes neuronales, en concreto, redes neuronales convolucionales, tenga gran dominio de la librería OpenCV y gran manejo en el lenguaje de programación Python.

\section{Timing}
\label{sec:timing-presupuesto}

Para calcular el presupuesto del proyecto es imprescindible conocer la duración del proyecto para tener en cuenta los costes debidos a la mano de obra.

Las fases necesarias para la realización del proyecto son las siguientes.

\begin{itemize}
    \item Formación inicial y revisión del Estado del Arte (1 mes)
  
    \begin{itemize}
        \item Recopilar información sobre el Estado del Arte: algoritmos, sistemas y bases de datos. (0,25 meses)
        \item Comparación entre diferentes estrategias de detección de objetos. (0,25 meses)
        \item Buscar implementaciones disponibles en el alcance del proyecto (0,25 meses)
        \item Estudiar las redes neuronales convolucionales (CNN) para identificar objetos en imágenes (0,25 meses)
    \end{itemize}
  
    \item Diseño, implementación y evaluación de algoritmos y librerías en la detección de personas y objetos (0,5 meses)
  
    \begin{itemize}
        \item Diseño, implementación/adaptación de la estrategia seleccionada para esta tarea (0,2 meses)
        \item Refinar el Estado del Arte (0,1 meses)
        \item Evaluación rigurosa de los algoritmos desarrollados en bases de datos relevantes (0,2 meses)
    \end{itemize}
  
    \item Diseño, implementación y evaluación de algoritmos y librerías en la seguimiento/rastreo de personas y objetos (1,5 meses)
  
    \begin{itemize}
        \item Diseño, implementación/adaptación de la estrategia seleccionada para esta tarea (0,7 meses)
        \item Refinar el Estado del Arte (0,2 meses)
        \item Evaluación rigurosa de los algoritmos desarrollados en bases de datos relevantes (0,6 meses)
    \end{itemize}

    \item Diseño, implementación y evaluación de algoritmos y librerías en la tarea de detección de objetos abandonados (1,5 meses):
  
    \begin{itemize}
        \item Diseño, implementación/adaptación de la estrategia seleccionada para esta tarea (0,7 meses)
        \item Refinar el Estado del Arte (0,2 meses)
        \item Evaluación rigurosa de los algoritmos desarrollados en bases de datos relevantes (0,6 meses)
    \end{itemize}
    
    \item Evaluación de la integración final de los algoritmos desarrollados en las bases de datos más relevantes (0,5 meses)
    
    \item Corrección de errores y comienzo de la redacción de la memoria del proyecto (1 mes)
    
    \item Finalización de la redacción de la memoria del proyecto (0,5 meses)
\end{itemize}

Cabe destacar que tareas como la de refinar y recopilar información el Estado del Arte, la comparación de diferentes estrategias de detección y rastreo o la evaluación de los algoritmos en las distintas bases de datos se repiten a lo largo de los 6,5 meses que dura el desarrollo del proyecto.

\section{Costes}
\label{sec:costes-presupuesto}

En los siguientes subapartados se va a calcular los costes asociados a la mano de obra necesaria en el desarrollo de cada una de las tareas descritas en la sección \ref{sec:timing-presupuesto}. También se tendrá en cuenta los costes asociados a los materiales \textit{hardware} y \textit{software} que componen los dos equipos utilizados para el desarrollo del proyecto. Por último se calculará el presupuesto total del proyecto teniendo en cuenta todo lo nombrado anteriormente.

\subsection{Costes mano de obra}
\label{subsec:costes-mano-obra}

La tabla \ref{tab:coste-mano-obra} presenta el desglose de las tareas descritas en la sección \ref{sec:timing-presupuesto} para calcular el coste de mano de obra que va a llevar cada una de las etapas.

\vspace{0.5cm}

\begin{table}[ht]
\centering
\caption{Costes de mano de obra}
\label{tab:coste-mano-obra}
\begin{tabular}{lccc}
\hline
\textbf{Concepto}                                     & \textbf{Horas} & \textbf{\begin{tabular}[c]{@{}c@{}}Coste Unitario\\ (\euro/h)\end{tabular}} & \textbf{\begin{tabular}[c]{@{}c@{}}Coste total\\ (\euro)\end{tabular}} \\ \hline
Formación inicial y revisión del Estado   del Arte    & 80             & 20                                                                    & 1.600                                                                        \\
D + I + E algoritmos detección objetos                & 40             & 20                                                                    & 800                                                                          \\
D + I + E algoritmos seguimiento objetos              & 120            & 20                                                                    & 2.400                                                                        \\
Desarrollo algoritmo detección objetos abandonados    & 120            & 20                                                                    & 2.400                                                                        \\
Evaluación algoritmos desarrollados en bases de datos & 40             & 20                                                                    & 800                                                                          \\
Corrección errores y comenzar a redactar la memoria   & 80             & 20                                                                    & 1.600                                                                        \\
Finalizar redacción de la memoria                     & 40             & 20                                                                    & 800                                                                          \\ \hline
\textbf{TOTAL}                                        & \textbf{520}   & \textbf{}                                                             & \textbf{\EUR{10.400}}                                                 
\end{tabular}
\end{table}
\subsection{Recursos hardware}
\label{subsec:recursos-hardware}

Los costes debidos a los recursos \textit{hardware} utilizados en el equipo A y el equipo B se visualizan en la siguiente tabla.

\vspace{0.5cm}

\begin{table}[ht]
\centering
\caption{Recursos hardware}
\label{tab:recursos-hardware}
\begin{tabular}{lccc}
\hline
\textbf{Concepto}            & \textbf{Unidades} & \textbf{Coste Unitario (\euro)} & \textbf{Coste total (\euro)} \\ \hline
Ordenador sobremesa equipo A & 1                 & 1.100                     & 1.100                              \\
Periféricos                  & 1                 & 250                       & 250                                \\
Monitor BenQ ZOWIE XL2411P   & 1                 & 199                       & 199                                \\
Ordenador portátil equipo B  & 1                 & 1.150                     & 1.150                              \\
Monitor BenQ GL2480          & 1                 & 129                       & 129                                \\ \hline
\textbf{TOTAL}               & \textbf{}         & \textbf{}                 & \textbf{\EUR{2.828}}        
\end{tabular}
\end{table}

\subsection{Recursos software}
\label{subsec:recursos-software}

Los costes debidos a los recursos \textit{software} utilizados en el equipo A y el equipo B se visualizan en la siguiente tabla.

\vspace{0.5cm}

\begin{table}[ht]
\centering
\caption{Recursos software}
\label{tab:recursos-software}
\begin{tabular}{lccc}
\hline
\multicolumn{1}{l}{\textbf{Concepto}} & \multicolumn{1}{c}{\textbf{Unidades}} & \multicolumn{1}{c}{\textbf{Coste Unitario (\euro)}} & \multicolumn{1}{c}{\textbf{Coste total (\euro)}} \\ \hline
Windows 10 Pro                        & 1                                     & 259                                           & 259                                                    \\
Microsoft Office 365                  & 1                                     & 49                                            & 49                                                     \\
TexMaker + MiKTeX-TeX                 & 1                                     & 0                                             & 0                                                      \\
Git                                   & 1                                     & 0                                             & 0                                                      \\
Ubuntu 18.04.4 LTS                    & 1                                     & 0                                             & 0                                                      \\
Anaconda Distribution                 & 1                                     & 0                                             & 0                                                      \\
Python 3.7.0                          & 1                                     & 0                                             & 0                                                      \\
R 3.6.3                               & 1                                     & 0                                             & 0                                                      \\ \hline
\textbf{TOTAL}                        & \textbf{}                             & \textbf{}                                     & \textbf{\EUR{308}}                              
\end{tabular}
\end{table}

\section{Presupuesto total}
\label{sec:presupuesto-total}

La tabla \ref{tab:presupuesto-total} recoge los costes totales del proyecto:

\vspace{0.5cm}

\begin{table}[ht]
\centering
\caption{Presupuesto total}
\label{tab:presupuesto-total}
\begin{tabular}{lc}
\hline
\textbf{Concepto}  & \textbf{Coste total (\euro)} \\ \hline
Coste mano de obra & 10.400                       \\ 
Recursos hardware  & 2.828                        \\
Recursos software  & 308                          \\ \hline

\textbf{TOTAL}     & \textbf{\EUR{13.536}}       
\end{tabular}
\end{table}

\vspace{0.5cm}

El presupuesto total del proyecto asciende a trece mil quinientos treinta y seis euros, IVA no incluido.
