
\chapter{Presupuesto}
\label{cha:presupuesto}

\section{Introducción}
\label{sec:intro-presupuesto}

En este apéndice se muestra el presupuesto del proyecto. Para ello, en los siguientes apartados se detallarán el perfil de personal necesario para desarollarlo, la duración estimada de cada una de las etapas que lo compone, de tal forma que se podrá calcular el coste de mano de obra de manera más precisa.

\section{Equipo de trabajo}
\label{sec:equipo-presupuesto}

\pendiente{Ver que ingeniero es el que se dedica a esto}Para la realización del proyecto se va a necesitar un Ingeniero de \textcolor{red}{Telecomunicaciones} que haya trabajado anteriormente con redes neuronales, y en concreto, con redes neuronales convolucionales, tenga gran dominio de la librería OpenCV y en el lenguaje de programación Python.

\section{Timing}
\label{sec:timing-presupuesto}

Las fases necesarias para la realización del desarrollo son las siguientes.

\begin{itemize}
\item formación inicial y revisión del Estado del Arte (1 mes)
  
  \begin{itemize}
    \item Recopilar información sobre el Estado del Arte: algoritmos, sistemas y bases de datos. (0,25 meses)
    \item Comparación entre diferentes estrategias de detección de objetos. (0,25 meses)
    \item Buscar implementaciones disponibles en el alcance del proyecto (0,25 meses)
    \item Estudiar las redes neuronales convolucionales (CNN) para identificar objetos en imágenes (0,25 meses)
  \end{itemize}
  
\item Diseño, implementación y evaluación de algoritmos y librerías en la detección de personas y objetos (0,5 meses)
  
  \begin{itemize}
  \item Diseño, implementación/adaptación de la estrategia seleccionada para esta tarea (0,2 meses)
  \item Refinar el Estado del Arte (0,1 meses)
  \item Evaluación rigurosa de los algoritmos desarrollados en bases de datos relevantes (0,2 meses)
  \end{itemize}
  
\item Diseño, implementación y evaluación de algoritmos y librerías en la seguimiento/rastero de personas y objetos (1,5 meses)
  
  \begin{itemize}
  \item Diseño, implementación/adaptación de la estrategia seleccionada para esta tarea (0,7 meses)
  \item Refinar el Estado del Arte (0,2 meses)
  \item Evaluación rigurosa de los algoritmos desarrollados en bases de datos relevantes (0,6 meses)
  \end{itemize}

\item Diseño, implementación y evaluación de algoritmos y bibliotecas en la tarea de detección de objetos abandonados (1,5 meses):
  
  \begin{itemize}
  \item Diseño, implementación/adaptación de la estrategia seleccionada para esta tarea (0,7 meses)
  \item Refinar el Estado del Arte (0,2 meses)
  \item Evaluación rigurosa de los algoritmos desarrollados en bases de datos relevantes (0,6 meses)
    \end{itemize}
    
\item Evaluación de la integración final de los algoritmos desarrollados en las bases de datos más relevantes. (0,5 meses)

\item Corrección de errores y comienzo de la redacción de la memoria del proyecto (1 mes)

\item Finalización de la redacción de la memoria del proyecto (0,5 meses)
  
\end{itemize}

Cabe destacar que tareas como la de refinar y recopilar información el Estado del Arte, la comparación de diferentes estrategias de detección y rastreo o la evaluación de los algortimos en las distintas bases de datos se repiten a lo largo de los 6,5 meses que dura el desarrollo del proyecto.

\section{Costes}
\label{sec:costes-presupuesto}

\section{Presupuesto total}
\label{sec:presupuesto-total}

Por todo lo anterior, supone una duración de 8 meses naturales con un coste de 12.000 \euro (sin IVA).
