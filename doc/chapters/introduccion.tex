
\chapter{Introducción}
\label{cha:introduccion}

En los últimos años \ldots

o

En este capítulo de introducción se quiere \ldots

\section{Motivación}
\label{sec:motivacion}

Introducción del proyecto haciendo alusiones a trabajos previos para poner en contexto, hablar sobre las técnicas empleadas y desarrollarlas brevemente para poner en contexto. Sin extenderse en en exceso ya que se desarrollará más en el apartado del Estado del Arte del capítulo 2 \ldots

\section{Objetivos}
\label{sec:objetivos}

El objetivo que se quiere llevar a cabo es el desarrollo de una estrategia de detección de objetos abandonados mediante el uso de aplicaciones de videovigilancia. En concreto se va a estudiar cuando se ha abandonado los siguientes tipos de objetos: mochilas, bolsos, maletines, bolsas de mano o maletas. Los espacios donde se va a evaluar la eficacia del sistema de detección desarrollado será tanto interiores como exteriores: estaciones de metro, centros comerciales, colegios/universidades o cualquier tipo de infraestructura que disponga de una o varias cámaras de videovigilancia. 

Los pasos para abordar este problema son los siguientes.

\begin{itemize}
    \item \textbf{Estudio del Estado de Arte actual}. Búsqueda y estudio de artículos referentes a la identificación de objetos abandonados en aplicaciones de videovigilancia dentro del Estado del Arte actual para tener un punto de partida. Por otro lado se deberá de buscar las bases de datos más relevantes en la evaluación de detección de objetos abandonados.
    \item \textbf{Evaluación de algoritmos algoritmos de detección de objetos más relevantes}. Se estudiará y comparará los algoritmos de detección de objetos actuales y se argumentará el motivo de la elección de uno concreto. Una vez seleccionado el algoritmo de detección se deberá de evaluar si trabajar sobre un conjunto de datos conocido o si por el contrario es interesante el entrenamiento de una red neuronal personalizada en la que se detecten solamente los objetos de interés. La elección del conjunto de datos de referencia para la evaluación del algoritmo de detección se decidirá teniendo en cuenta las principales métricas de clasificación de \textit{Machine Learning} así como la matriz de confusión. Teniendo un conjunto de datos de referencia se ejecutará el algoritmo en las bases de datos más utilizadas para evaluar las predicciones.
    \item \textbf{Evaluación de algoritmos de seguimiento o \textit{tracking} de objetos más relevantes}. En base al modelo del algoritmo de detección de objetos seleccionado se estudiará y evaluará los algoritmos de seguimiento actuales. El objetivo de este punto es que en la detección de objetos y personas, cada elemento tenga una identidad propia a lo largo del tiempo, o lo que es lo mismo, a lo largo de los frames de un video. De tal manera que, cuando se implemente el algoritmo de detección de objetos abandonados sea más sencillo la asociación de persona-objeto. De igual manera que en el algoritmo de detección, también se ejecutará el algoritmo en las bases de datos más relevantes en detección de objetos abandonados para evaluar si \textit{tracking} sobre personas y objetos de interés a lo largo de un video.
    \item \textbf{Implementación y evaluación de un algoritmo de detección de objetos abandonados}. Se desarrollará un algoritmo capaz de determinar si un objeto ha sido abandonado o no. Existen tres posibles escenarios. El primero es que el objeto se encuentre móvil durante toda la ejecución del video y no se pueda asociar a ninguna persona como propietario. La segunda es que a una persona a la que se le ha asociado un objeto se alejen más de una cierta distancia a lo largo de un número determinado frames. La tercera es que a una persona a la que se le ha asociado un objeto desaparezca y se este detectando el únicamente el objeto durante un número determinado de frames. Para estos dos últimos casos se deberá de establecer una asociación persona-objeto y estudiar su comportamiento a lo largo del video.
\end{itemize}

\section{Estructura de la memoria}
\label{sec:estructura-memoria}
En este apartado se resume brevemente como se encuentra organizados los contenidos que componen el presente Trabajo Fin de Máster.

\begin{itemize}
    \item \textbf{Capítulo \ref{cha:introduccion}: Introducción.} Se expondrá la motivación que ha impulsado la realización de este Trabajo Fin de Máster. Se citará brevemente trabajos previos que han servido de esqueleto del proyecto. Por otro lado se argumentarán los objetivos que se pretenden alcanzar.
    \item \textbf{Capítulo \ref{cha:estudio-teorico}: Estudio teórico.} Se realizará un estudio exhaustivo del Estado del Arte en lo referente a algoritmos de detección de objetos abandonados en aplicaciones de videovigilancia. Se evaluará y comparará los algoritmos de detección de objetos y algoritmos de seguimiento más relevantes y se desarrollarán en más profundidad los seleccionados argumentando los motivos. 
    \item \textbf{Capítulo \ref{cha:desarrollo}: Desarrollo algoritmo de detección de objetos abandonados.} Se desarrollará un algoritmo de detección de objetos abandonados en el que se tendrá que tener en cuenta si el objeto tiene o no propietario y en el caso de que lo tenga, crear una asociación persona-objeto para evaluar cuando se produce el abandono del objeto.
    \item \textbf{Capítulo \ref{cha:resultados}: Resultados.} Se expondrán los resultados obtenidos tras el desarrollo del proyecto. Se describirá las bases de datos utilizadas para la evaluación de los algoritmos así como las métricas de calidad más relevantes en cada algoritmo.
    \item \textbf{Capítulo \ref{cha:concl-lineas-futuras}: Conclusiones y líneas futuras.} Se expondrán las conclusiones que se han llegado al finalizar este proyecto. Se explicarán las ventajas y limitaciones que presenta la idea propuesta para su desarrollo. Por otro lado se argumentarán posibles vías de desarrollo derivados de este proyecto, así como nuevos proyectos donde se puedan emplear el mismo algoritmo de detección y seguimiento y solamente se tenga que programar un algoritmo que realice una función concreta.
    \item \textbf{Bibliografía.} Se incluye cada uno de los artículos, repositorios, conjuntos de datos y toda clase de material consultado para la elaboración de este Trabajo Fin de Máster indicando la fecha de último acceso. 
    \item \textbf{Apéndice \ref{cha:pliego-de-condiciones}.} Se hace referencia al pliego de condiciones donde se tendrán en cuenta las especificaciones \textit{hardware} y \textit{software} que se han empleado en el desarrollo de este proyecto.
    \item \textbf{Apéndice \ref{cha:presupuesto}.} Se muestra el presupuesto donde se incluye los costes materiales \textit{hardware} y \textit{software} y el coste de la mano de obra en función a la duración estimada del proyecto.
    \item \textbf{Apéndice \ref{cha:manual-usuario-instalacion}.} Se detallan cada unos de los pasos necesarios para instalar todas las dependencias necesarias para el funcionamiento de los algoritmos. Una vez instalado todo el \textit{software} y librerías se puede consultar el manual de usuario donde se indica como poner en funcionamiento cualquiera de los algoritmos desarrollados en este proyecto.
\end{itemize}




