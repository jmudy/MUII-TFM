
\chapter{Resultados}
\label{cha:resultados}

Aquí se mostrarán los resultados del proyecto \ldots

Como dijo \cite{lin2015microsoft} \ldots

\section{Introducción}
\label{sec:intro-resultados}

\section{Entorno experimental}
\label{sec:desarrollo-resultados}

\newpage

\subsection{Bases de datos utilizadas}
\label{subsec:bases-datos}

\subsubsection{PETS 2007 Dataset}

\begin{figure}[ht]
  \centering
  \begin{subfigure}[b]{0.4\textwidth}
    \includegraphics[width=\textwidth]{img/chapters/resultados/bases-datos/pets2007_1.jpeg}
    \caption{}
    \label{fig:pets2007_1}
  \end{subfigure}
  \qquad\qquad
  \begin{subfigure}[b]{0.4\textwidth}
    \includegraphics[width=\textwidth]{img/chapters/resultados/bases-datos/pets2007_2.jpeg}
    \caption{}
    \label{fig:pets2007_2}
  \end{subfigure}
  \caption{Imágenes extraídas del dataset PETS2007 \cite{pets2007-dataset}.
    (\protect\subref{fig:pets2007_1}) Frame donde un individuo deja su bolsa de equipaje sobre el suelo.
    (\protect\subref{fig:pets2007_2}) Otro frame en el cual una persona roba el equipaje el individuo.}
  \label{fig:pets2007}
\end{figure}

\subsubsection{AVSS AB 2007 Dataset}

\begin{figure}[ht]
  \centering
  \begin{subfigure}[b]{0.4\textwidth}
    \includegraphics[width=\textwidth]{img/chapters/resultados/bases-datos/AVSSAB_1.jpg}
    \caption{}
    \label{fig:avssab2007_1}
  \end{subfigure}
  \qquad\qquad
  \begin{subfigure}[b]{0.4\textwidth}
    \includegraphics[width=\textwidth]{img/chapters/resultados/bases-datos/AVSSAB_2.jpg}
    \caption{}
    \label{fig:avssab2007_2}
  \end{subfigure}
  \caption{Imágenes extraídas del dataset AVSS AB 2007 \cite{AVSSAB2007-dataset}.
    (\protect\subref{fig:avssab2007_1}) Frame donde un individuo abandona su maleta en el andén.
    (\protect\subref{fig:avssab2007_2}) Otro frame donde una bolsa de equipaje está perdida en el andén durante todo el video.}
  \label{fig:avssab2007}
\end{figure}

\newpage

\subsubsection{GBA 2018 Dataset}

\begin{figure}[ht!]
  \centering
  \begin{subfigure}[b]{0.45\textwidth}
    \includegraphics[width=\textwidth]{img/chapters/resultados/bases-datos/GBA_1.jpg}
    \caption{}
    \label{fig:GBA_1}
  \end{subfigure}
  \qquad\qquad
  \begin{subfigure}[b]{0.45\textwidth}
    \includegraphics[width=\textwidth]{img/chapters/resultados/bases-datos/GBA_2.jpg}
    \caption{}
    \label{fig:GBA_2}
  \end{subfigure}
  \caption{Imágenes extraídas del dataset GBA 2018 \cite{gba-dataset}.
    (\protect\subref{fig:GBA_1}) Frame donde una bolsa de mano es abandonada en el pasillo.
    (\protect\subref{fig:GBA_2}) Otro frame donde varias bolsas y maletas están alejadas de sus propietarios.}
  \label{fig:GBA}
\end{figure}

\subsubsection{ABODA Dataset}

\begin{figure}[ht]
  \centering
  \begin{subfigure}[b]{0.4\textwidth}
    \includegraphics[width=\textwidth]{img/chapters/resultados/bases-datos/aboda_1.jpg}
    \caption{}
    \label{fig:aboda_1}
  \end{subfigure}
  \qquad\qquad
  \begin{subfigure}[b]{0.4\textwidth}
    \includegraphics[width=\textwidth]{img/chapters/resultados/bases-datos/aboda_2.jpg}
    \caption{}
    \label{fig:aboda_2}
  \end{subfigure}
  \caption{Imágenes extraídas del dataset ABODA \cite{aboda-dataset}.
    (\protect\subref{fig:aboda_1}) Frame donde dos individuos se alejan de una mochila.
    (\protect\subref{fig:aboda_2}) Otro frame donde un individuo abandona una bolsa en la mesa.}
  \label{fig:aboda}
\end{figure}

\subsubsection{MS COCO 2017 Dataset}
\label{subsubsec:coco-dataset}

\textcolor{red}{La base de datos MS COCO 2017 es gran conjunto de datos que contiene más de 200 000 imágenes distribuidas en 80 clases de objetos que representan escenas del mundo real. Es una base de datos lo suficientemente grande para que una red bien entrenada con este conjunto de datos sea capaz de aprender características visuales de calidad para el reconocimiento y detección de objetos en imágenes.}

\begin{figure}[ht]
\centering
\includegraphics[width=0.9\textwidth]{img/chapters/resultados/bases-datos/cocodataset.png}
\caption{\label{fig:cocodataset}Categorías de objetos del dataset MS COCO 2017}
\end{figure}

\newpage

\subsubsection{Open Images Dataset v4}
\label{subsubsec:OIDv4-dataset}

\begin{figure}[ht]
\centering
\includegraphics[width=0.7\textwidth]{img/chapters/resultados/bases-datos/oid-classes.png}
\caption{\label{fig:oiddataset}Categorías de objetos del dataset Open Images Dataset v4}
\end{figure}

\subsection{Métricas de calidad}
\label{subsec:metricas-calidad}

\textcolor{red}{Ver TFM de Ana Ruipérez Gómez o Pedro López Miguel para ver como ha redactado las métricas} \pendiente{ojo a esto!!!}

En esta sección se va a exponer las distintas métricas \cite{padillaCITE2020} que han sido utilizadas para evaluar el conjunto de datos utilizado para el entrenamiento de la red neuronal.

\subsubsection{Intersección sobre la unión (IoU)}
\label{subsubsec:iou}

Intersection over Union (IoU) es una medida basada en el índice Jaccard que evalúa la superposición entre dos cuadros delimitadores. Requiere un cuadro delimitador de verdad del terreno $B_{gt}$ y un cuadro delimitador previsto $B_{p}$. Aplicando el IoU podemos saber si una detección es válida (verdadero positivo) o no (falso positivo).

El IoU viene dado por el área de superposición entre el cuadro delimitador predicho y el cuadro delimitador de verdad del suelo dividido por el área de unión entre ellos:

\begin{equation}
\label{eq:iou}
\text{IoU}=\frac{\text{area}\left(B_{p} \cap B_{gt} \right)}{\text{area}\left(B_{p} \cup B_{gt} \right)}    
\end{equation}

La siguiente imagen ilustra el IoU entre un cuadro delimitador de verdad del terreno (en verde) y un cuadro delimitador detectado (en rojo).

\begin{figure}[ht]
\centering
\includegraphics[width=0.65\textwidth]{img/chapters/resultados/metricas/iou.png}
\caption{\label{fig:iou}Área de superposición IoU entre los cuadros delimitadores}
\end{figure}

\subsubsection{TP, TN, FP y FN}
\label{subsubsec:tp+tn+fp+fn}

Otros parámetros básicos en las métricas de calidad que conforman la matriz de confusión \cite{confusion-matrix} son:

\begin{itemize}
    \item \textbf{Verdadero positivo (TP)}: Número de predicciones donde el clasificador predice correctamente la clase positiva como positiva. IoU $\geqslant$ \textit{threshold}
    \item \textbf{Verdadero negativo (TN)}: Número de predicciones donde el clasificador predice correctamente la clase negativa como negativa. No se utiliza en el cálculo de métricas.
    \item \textbf{Falso positivo (FP)}: Número de predicciones donde el clasificador predice incorrectamente la clase negativa como positiva. IoU <\ \textit{threshold}
    \item \textbf{Falso negativo (FN)}: Número de predicciones donde el clasificador predice incorrectamente la clase positiva como negativa.
\end{itemize}

Típicamente el \textit{threshold} toma valores del 50\%, 75\% o 95\%.

\begin{figure}[ht]
\centering
\includegraphics[width=0.45\textwidth]{img/chapters/resultados/metricas/confusion-matrix.png}
\caption{\label{fig:confusion-matrix}Matriz de confusión}
\end{figure}

\subsubsection{Precisión}
\label{subsubsec:precision}

La precisión es la capacidad de un modelo para identificar solo los objetos relevantes. Es el porcentaje de predicciones positivas correctas y viene dado por la siguiente expresión \ref{eq:precision}:

\begin{equation}
\label{eq:precision}
\text{P} = \frac{\text{TP}}{\text{TP}+\text{FP}}=\frac{\text{TP}}{\text{all detections}}
\end{equation}

\subsubsection{Exhaustividad}
\label{subsubsec:exhaustividad}

La exhaustividad es la capacidad de un modelo para encontrar todos los casos relevantes (todos los cuadros delimitadores de verdad del terreno). Es el porcentaje de verdadero positivo detectado entre todas las verdades fundamentales relevantes y viene dado por la siguiente expresión:

\begin{equation}
\label{eq:exhaustividad}
\text{R} = \frac{\text{TP}}{\text{TP}+\text{FN}}=\frac{\text{TP}}{\text{all ground truths}}
\end{equation}

\subsubsection{Valor-F}
\label{subsubsec:valor-f}

 El Valor-F se trata de una medida estadística de precisión muy utilizado en las pruebas test de algoritmos. Es la media armónica que combina los valores de la precisión y la exhaustividad. Viene dado por la expresión \ref{eq:valor-f}:

\begin{equation}
\label{eq:valor-f}
\text{F} = \frac{2 \cdotp \text{P} \cdotp \text{R}}{\text{P}+\text{R}}
\end{equation}

\textcolor{red}{Poner un ejemplo sencillo para que se entienda}

\subsubsection{Precisión media}
\label{subsubsec:averageprecision}

La precisión media es el valor medio de 11 puntos en la curva P-R para cada posible umbral (cada probabilidad de detección) para la misma clase (Precisión-Exhaustividad). En la ecuación \ref{eq:ap} se muestra el cálculo de la precisión media:

\begin{equation}
\label{eq:ap}
\text{AP}=\frac{1}{11} \sum_{r\in \left \{ 0, 0.1, ...,1 \right \}}\rho_{\text{interp}\left ( r \right )}
\end{equation}

con

$$\rho_{\text{interp}} = \max_{\tilde{r}:\tilde{r} \geq r} \rho\left ( \tilde{r} \right )$$

donde $\rho\left ( \tilde{r} \right )$ es la precisión medida en la exhaustividad $\tilde{r}$

Por otro lado, el mAP es la media de los AP de todas las categorías de objetos. El mAP se representa mediante la siguiente ecuación:

\begin{equation}
\label{eq:map}
\text{mAP} = \frac{1}{N} \sum_{i=1}^{N} \text{AP}_{1}
\end{equation}

\subsection{Estrategia y metodología de experimentación}
\label{subsec:estrategia-metodologia}

\subsubsection{Entrenamiento con Open Image Dataset v4}
\label{subsubsec:train-openimagesv4}

\textcolor{red}{Aquí explicar como he entrenado una red neuronal con otro dataset a partir del repositorio \cite{OIDv4_ToolKit}}

\textcolor{red}{Explicar que se ha tomado 1500 imágenes de entrenamiento de las clases: person, handbag, backpack, suitcase y 300 imágenes de validación.}

\vspace{0.5cm}
\begin{lstlisting}[language=iPython,caption=Descarga set de datos Open Images Dataset v4,captionpos=b,label={lst:download-oidv4}]
# Clonar el repositorio de Github
git clone https://github.com/theAIGuysCode/OIDv4_ToolKit.git
cd OIDv4_ToolKit

# Instalacion de las librerias y dependendencias
pip install -r requirements.txt

# Descarga de las imagenes de entrenamiento con un limite de 1500
python main.py downloader --classes Person Handbag Backpack Suitcase --type_csv train --limit 1500 --multiclasses 1

# Descarga de las imagenes de validacion con un limite de 300
python main.py downloader --classes Person Handbag Backpack Suitcase --type_csv validation --limit 300 --multiclasses 1

# Convertir etiquetas al formato de Darknet
python convert_annotations.py

\end{lstlisting}

\begin{figure}[ht]
\centering
\includegraphics[width=0.3\textwidth]{img/chapters/resultados/bases-datos/download-oidv4.png}
\caption{\label{fig:download-oidv4}Descarga del set de datos Open Images Dataset v4}
\end{figure}

\begin{figure}[ht]
\centering
\includegraphics[width=0.3\textwidth]{img/chapters/resultados/bases-datos/bbox-oidv4.png}
\caption{\label{fig:bbox-oidv4}Estructura de las etiquetas de Open Images Dataset v4}
\end{figure}

La estructura que siguen las etiquetas del set de datos de Open Images Dataset v4 es la siguiente:

\texttt{nombre de la clase x left top y x right bottom y}

\subsubsection{Métricas de calidad en Open Image Dataset v4}
\label{subsubsec:metricas-calidad-openimagesv4}

Tras entrenar YOLOv4 con el conjunto de datos personalizado que se ha explicado en el apartado anterior \ref{subsubsec:train-openimagesv4} se va a evaluar las métricas de calidad. Gracias al framework Darknet \cite{darknet13} es fácil poder evaluar las métricas aplicando el siguiente comando en el terminal:

\vspace{0.5cm}
\begin{lstlisting}[language=iPython,caption=Evaluación métricas de calidad del conjunto de datos utilizado para el entrenamiento de la red neuronal de detección de objetos,captionpos=b,label={lst:darknet-map}]
# Evaluacion de metricas de interes
./darknet detector train data/obj.data cfg/yolov4-obj.cfg yolov4.conv.137 -dont_show -map
\end{lstlisting}

\begin{figure}[ht]
\centering
\includegraphics[width=0.3\textwidth]{img/chapters/resultados/metricas/chart_train.png}
\caption{\label{fig:chart-train}Evolución del mAP y pérdidas a lo largo de las interaciones durante el entrenamiento de la red neuronal con el set de datos de OIDv4}
\end{figure}

\begin{figure}[ht]
\centering
\includegraphics[width=0.3\textwidth]{img/chapters/resultados/metricas/metrics-during-training.png}
\caption{\label{fig:metrics-during-train}Métricas durante el entrenamiento de la red neuronal con el set de datos de OIDv4}
\end{figure}

En la tabla \ref{tab:metricas-test1_1}, \ref{tab:metricas-test1_2}, \ref{tab:metricas-test1_3} y \ref{tab:metricas-test1_4} se reflejan las métricas más relevantes cada 1000 iteraciones del entrenamiento de la red neuronal.

\newpage

\begin{table}[ht]
\centering
\caption{Métricas de calidad en el primer entrenamiento con OIDv4 [1]}
\label{tab:metricas-test1_1}
\begin{tabular}{lcccc}
\hline
\textbf{Iterations} & \textbf{\begin{tabular}[c]{@{}c@{}}AP person\\ (\%)\end{tabular}} & \textbf{\begin{tabular}[c]{@{}c@{}}AP handbag\\ (\%)\end{tabular}} & \textbf{\begin{tabular}[c]{@{}c@{}}AP backpack\\ (\%)\end{tabular}} & \textbf{\begin{tabular}[c]{@{}c@{}}AP suitcase\\ (\%)\end{tabular}} \\ \hline
1.000               & 31,26                                                             & 85,45                                                              & 67,99                                                               & 42,15                                                               \\
\textbf{2.000}      & \textbf{43,96}                                                    & \textbf{92,39}                                                     & \textbf{63,88}                                                      & \textbf{64,79}                                                      \\
3.000               & 36,16                                                             & 90,10                                                              & 67,88                                                               & 53,33                                                               \\
4.000               & 35,56                                                             & 91,99                                                              & 64,61                                                               & 57,74                                                               \\
5.000               & 34,21                                                             & 87,35                                                              & 68,73                                                               & 48,77                                                               \\
6.000               & 36,74                                                             & 89,48                                                              & 65,83                                                               & 49,09                                                               \\
7.000               & 34,76                                                             & 88,94                                                              & 68,20                                                               & 51,24                                                               \\
8.000               & 38,30                                                             & 87,69                                                              & 72,00                                                               & 58,89                                                               \\ \hline
\end{tabular}
\end{table}

\begin{table}[ht]
\centering
\caption{Métricas de calidad en el primer entrenamiento con OIDv4 [2]}
\label{tab:metricas-test1_2}
\begin{tabular}{lcccc}
\hline
\textbf{Iterations} & \textbf{TP person} & \textbf{TP handbag} & \textbf{TP backpack} & \textbf{TP suitcase} \\ \hline
1.000               & 249                & 52                  & 19                   & 16                   \\
\textbf{2.000}      & \textbf{338}       & \textbf{54}         & \textbf{19}          & \textbf{19}          \\
3.000               & 324                & 54                  & 21                   & 18                   \\
4.000               & 329                & 54                  & 20                   & 20                   \\
5.000               & 313                & 50                  & 23                   & 16                   \\
6.000               & 323                & 52                  & 20                   & 13                   \\
7.000               & 288                & 53                  & 19                   & 18                   \\
8.000               & 308                & 49                  & 21                   & 19                   \\ \hline
\end{tabular}
\end{table}

\begin{table}[ht]
\centering
\caption{Métricas de calidad en el primer entrenamiento con OIDv4 [3]}
\label{tab:metricas-test1_3}
\begin{tabular}{lcccc}
\hline
\textbf{Iterations} & \textbf{FP person} & \textbf{FP handbag} & \textbf{FP backpack} & \textbf{FP suitcase} \\ \hline
1.000               & 416                & 20                  & 11                   & 42                   \\
\textbf{2.000}      & \textbf{489}       & \textbf{15}         & \textbf{16}          & \textbf{7}           \\
3.000               & 479                & 22                  & 18                   & 36                   \\
4.000               & 521                & 10                  & 20                   & 19                   \\
5.000               & 474                & 14                  & 19                   & 22                   \\
6.000               & 446                & 7                   & 14                   & 13                   \\
7.000               & 391                & 19                  & 13                   & 16                   \\
8.000               & 412                & 11                  & 16                   & 17                   \\ \hline
\end{tabular}
\end{table}

\begin{table}[ht!]
\centering
\caption{Métricas de calidad en el primer entrenamiento con OIDv4 [4]}
\label{tab:metricas-test1_4}
\begin{tabular}{lcccccccc}
\hline
\textbf{Iterations} & \textbf{TP}  & \textbf{FP}  & \textbf{FN}  & \textbf{\begin{tabular}[c]{@{}c@{}}Precision\\ (\%)\end{tabular}} & \textbf{\begin{tabular}[c]{@{}c@{}}Recall\\ (\%)\end{tabular}} & \textbf{\begin{tabular}[c]{@{}c@{}}F-score\\ (\%)\end{tabular}} & \textbf{\begin{tabular}[c]{@{}c@{}}Average IoU\\ (\%)\end{tabular}} & \textbf{\begin{tabular}[c]{@{}c@{}}mAP @ 0.5\\ (\%)\end{tabular}} \\ \hline
1.000               & 336          & 489          & 367          & 40,73                                                             & 47,80                                                          & 43,98                                                           & 29,36                                                               & 56,71                                                             \\
\textbf{2.000}      & \textbf{430} & \textbf{527} & \textbf{273} & \textbf{44,93}                                                    & \textbf{61,17}                                                 & \textbf{51,81}                                                  & \textbf{34,13}                                                      & \textbf{66,25}                                                    \\
3.000               & 417          & 555          & 286          & 42,90                                                             & 59,32                                                          & 49,79                                                           & 32,29                                                               & 61,87                                                             \\
4.000               & 423          & 570          & 280          & 42,60                                                             & 60,17                                                          & 49,88                                                           & 33,30                                                               & 62,47                                                             \\
5.000               & 402          & 529          & 301          & 43,18                                                             & 57,18                                                          & 49,20                                                           & 32,53                                                               & 59,77                                                             \\
6.000               & 408          & 480          & 295          & 45,95                                                             & 58,04                                                          & 51,29                                                           & 36,76                                                               & 60,28                                                             \\
7.000               & 378          & 439          & 325          & 46,27                                                             & 53,77                                                          & 49,74                                                           & 36,90                                                               & 60,78                                                             \\
8.000               & 397          & 456          & 306          & 46,54                                                             & 56,47                                                          & 51,03                                                           & 37,69                                                               & 64,22                                                             \\ \hline
\end{tabular}
\end{table}

\newpage

\textcolor{red}{Aquí explicar porque en función de las métricas obtenidas no es un buen modelo y se debe de reentrenar la red. El IoU sale muy bajo <\ 50\%, por tanto salen muchos FP}

\begin{table}[ht]
\centering
\caption{Métricas de calidad en el segundo entrenamiento con OIDv4 [1]}
\label{tab:metricas-test2_1}
\begin{tabular}{lcccccc}
\hline
\textbf{Iterations} & \textbf{\begin{tabular}[c]{@{}c@{}}AP person\\ (\%)\end{tabular}} & \textbf{\begin{tabular}[c]{@{}c@{}}AP bags\\ (\%)\end{tabular}} & \textbf{TP person} & \textbf{TP bags} & \textbf{FP person} & \textbf{FP bags} \\ \hline
1.000               & 30,81                                                             & 21,61                                                           & 5.216              & 231               & 9.220              & 579              \\
2.000               & 38,38                                                             & 53,59                                                           & 6.542              & 362               & 11.789             & 354              \\
3.000               & 33,51                                                             & 68,56                                                           & 7.232              & 419               & 18.887             & 411              \\
4.000               & 41,31                                                             & 77,12                                                           & 7.105              & 427               & 11.397             & 222              \\
5.000               & 38,86                                                             & 75,78                                                           & 6.586              & 444               & 11.735             & 398              \\
6.000               & 36,29                                                             & 66,49                                                           & 6.556              & 426               & 12.506             & 537              \\
7.000               & 39,94                                                             & 67,78                                                           & 6.246              & 418               & 9.744              & 523              \\
8.000               & 31,69                                                             & 69,07                                                           & 6.082              & 417               & 13.353             & 422              \\
9.000               & 43,34                                                             & 78,37                                                           & 6.773              & 451               & 9.846              & 373              \\
\textbf{10.000}     & \textbf{43,40}                                                    & \textbf{78,53}                                                  & \textbf{6.174}     & \textbf{426}      & \textbf{7.149}     & \textbf{217}     \\
11.000              & 38,81                                                             & 76,60                                                           & 7.166              & 446               & 12.162             & 387              \\
12.000              & 41,72                                                             & 78,10                                                           & 6.926              & 444               & 10.387             & 289              \\
13.000              & 39,48                                                             & 74,67                                                           & 6.575              & 406               & 9.850              & 238              \\
14.000              & 41,85                                                             & 73,69                                                           & 6.844              & 432               & 10.092             & 385              \\
15.000              & 40,19                                                             & 75,37                                                           & 6.915              & 423               & 11.426             & 252              \\
16.000              & 41,26                                                             & 75,17                                                           & 6.661              & 423               & 9.330              & 219              \\
17.000              & 42,13                                                             & 78,77                                                           & 6.991              & 433               & 9.924              & 242              \\
18.000              & 40,97                                                             & 75,45                                                           & 6.871              & 432               & 10.243             & 300              \\
19.000              & 39,01                                                             & 73,23                                                           & 6.822              & 428               & 10.791             & 333              \\
20.000              & 41,37                                                             & 78,38                                                           & 7.011              & 432               & 10.103             & 232              \\ \hline
\end{tabular}
\end{table}

\begin{table}[ht!]
\centering
\caption{Métricas de calidad en el segundo entrenamiento con OIDv4 [2]}
\label{tab:metricas-test2_2}
\begin{tabular}{lcccccccc}
\hline
\textbf{Iterations} & \textbf{TP}    & \textbf{FP}    & \textbf{FN}    & \textbf{\begin{tabular}[c]{@{}c@{}}Precision\\ (\%)\end{tabular}} & \textbf{\begin{tabular}[c]{@{}c@{}}Recall\\ (\%)\end{tabular}} & \textbf{\begin{tabular}[c]{@{}c@{}}F-score\\ (\%)\end{tabular}} & \textbf{\begin{tabular}[c]{@{}c@{}}Average\\ IoU (\%)\end{tabular}} & \textbf{\begin{tabular}[c]{@{}c@{}}mAP @ 0.5\\ (\%)\end{tabular}} \\ \hline
1.000               & 5.447          & 9.799          & 6.379          & 35,73                                                             & 46,06                                                          & 40,24                                                           & 25,72                                                               & 26,21                                                             \\
2.000               & 6.904          & 12.143         & 4.922          & 36,25                                                             & 58,38                                                          & 44,73                                                           & 27,30                                                               & 45,98                                                             \\
3.000               & 7.651          & 19.298         & 4.175          & 28,39                                                             & 64,70                                                          & 39,46                                                           & 21,66                                                               & 51,04                                                             \\
4.000               & 7.532          & 11.619         & 4.294          & 39,33                                                             & 63,69                                                          & 48,63                                                           & 30,90                                                               & 59,22                                                             \\
5.000               & 7.030          & 12.133         & 4.796          & 36,69                                                             & 59,45                                                          & 45,37                                                           & 28,28                                                               & 57,32                                                             \\
6.000               & 6.982          & 13.043         & 4.844          & 34,87                                                             & 59,04                                                          & 43,84                                                           & 27,23                                                               & 51,39                                                             \\
7.000               & 6.664          & 10.267         & 5.162          & 39,36                                                             & 56,35                                                          & 46,35                                                           & 30,96                                                               & 53,86                                                             \\
8.000               & 6.499          & 13.775         & 5.327          & 32,06                                                             & 54,96                                                          & 40,49                                                           & 24,96                                                               & 50,38                                                             \\
9.000               & 7.224          & 10.219         & 4.602          & 41,41                                                             & 61,09                                                          & 49,36                                                           & 32,95                                                               & 60,85                                                             \\
\textbf{10.000}     & \textbf{6.600} & \textbf{7.366} & \textbf{5.226} & \textbf{47,26}                                                    & \textbf{55,81}                                                 & \textbf{51,18}                                                  & \textbf{37,83}                                                      & \textbf{60,97}                                                    \\
11.000              & 7.612          & 12.549         & 4.214          & 37,76                                                             & 64,37                                                          & 47,59                                                           & 30,20                                                               & 57,70                                                             \\
12.000              & 7.370          & 10.676         & 4.456          & 40,84                                                             & 62,32                                                          & 49,34                                                           & 32,34                                                               & 59,91                                                             \\
13.000              & 6.981          & 10.088         & 4.845          & 40,90                                                             & 59,03                                                          & 48,32                                                           & 32,83                                                               & 57,07                                                             \\
14.000              & 7.276          & 10.477         & 4.550          & 40,98                                                             & 61,53                                                          & 49,20                                                           & 32,61                                                               & 57,77                                                             \\
15.000              & 7.338          & 11.678         & 4.488          & 38,59                                                             & 62,05                                                          & 47,58                                                           & 30,53                                                               & 57,78                                                             \\
16.000              & 7.084          & 9.549          & 4.742          & 42,59                                                             & 59,90                                                          & 49,78                                                           & 33,87                                                               & 58,22                                                             \\
17.000              & 7.424          & 10.166         & 4.402          & 42,21                                                             & 62,78                                                          & 50,48                                                           & 34,53                                                               & 60,45                                                             \\
18.000              & 7.303          & 10.543         & 4.523          & 40,92                                                             & 61,75                                                          & 49,22                                                           & 33,11                                                               & 58,21                                                             \\
19.000              & 7.250          & 11.124         & 4.576          & 39,46                                                             & 61,31                                                          & 48,01                                                           & 31,90                                                               & 56,12                                                             \\
20.000              & 7.443          & 10.335         & 4.383          & 41,87                                                             & 62,94                                                          & 50,28                                                           & 34,35                                                               & 59,93                                                             \\ \hline
\end{tabular}
\end{table}

\newpage

\subsubsection{Métricas de calidad en MS COCO 2017 Dataset}
\label{subsubsec:metricas-calidad-coco}

\begin{table}[ht]
\centering
\caption{Comparativa métricas de calidad entre los test en OIDv4 y MS COCO 2017 [1]}
\label{tab:comparativa-metricas1}
\begin{tabular}{lccc}
\hline
\textbf{Dataset}                   & \textbf{TP}          & \textbf{FP}          & \textbf{FN}          \\ \hline
\textbf{MS COCO 2017}              & \textbf{22.730}      & \textbf{10.889}      & \textbf{13.027}      \\
OIDv4 test 1                       & 430                  & 527                  & 273                \\
OIDv4 test 2                       & 6.600                & 7.366                & 5.226                \\ \hline
\end{tabular}
\end{table}

\begin{table}[ht]
\centering
\caption{Comparativa métricas de calidad entre los dos test en OIDv4 y MS COCO 2017 [2]}
\label{tab:comparativa-metricas2}
\begin{tabular}{cccccc}
\hline
\rowcolor[HTML]{FFFFFF} 
\textbf{Dataset}             & \textbf{\begin{tabular}[c]{@{}c@{}}Precision\\ (\%)\end{tabular}} & \textbf{\begin{tabular}[c]{@{}c@{}}Recall\\ (\%)\end{tabular}} & \textbf{\begin{tabular}[c]{@{}c@{}}F-score\\ (\%)\end{tabular}} & \textbf{\begin{tabular}[c]{@{}c@{}}average IoU\\ (\%)\end{tabular}} & \textbf{\begin{tabular}[c]{@{}c@{}}mAP @ 0.5\\ (\%)\end{tabular}} \\ \hline
\textbf{MS COCO 2017}        & \textbf{67,61}                                                    & \textbf{63,57}                                                 & \textbf{65,53}                                                  & \textbf{56.04}                                                      & \textbf{64.16}                                                    \\
OIDv4 test 1                 & 44,93                                                             & 61,17                                                          & 51,81                                                           & 34,13                                                               & 66,25                                                             \\
OIDv4 test 2                 & 47,26                                                             & 55,81                                                          & 51,18                                                           & 37,83                                                               & 60,97                                                             \\ \hline
\end{tabular}
\end{table}

\begin{table}[ht]
\centering
\caption{Métricas de calidad de MS COCO 2017 en las clases de interés}
\label{tab:metricas-clases-coco}
\begin{tabular}{lccc}
\hline
\textbf{Class} & \textbf{AP(\%)} & \textbf{TP} & \textbf{FP} \\ \hline
Person         & 79,53           & 7.923       & 3.168       \\
Backpack       & 44,10           & 172         & 156         \\
Handbag        & 29,83           & 158         & 215         \\
Suitcase       & 71,08           & 205         & 102         \\ \hline
\end{tabular}
\end{table}

\newpage

\begin{figure}[ht]
\centering
\includegraphics[width=1\textwidth]{img/chapters/resultados/metricas/metrics-train1.png}
\caption{\label{fig:metrics-train1}Resumen métricas primer entrenamiento de la red neuronal con el dataset de OIDv4}
\end{figure}

\newpage

\begin{figure}[ht]
\centering
\includegraphics[width=1\textwidth]{img/chapters/resultados/metricas/metrics-train2.png}
\caption{\label{fig:metrics-train2}Resumen métricas segundo entrenamiento de la red neuronal con el dataset de OIDv4}
\end{figure}


\section{Resultados experimentales}
\label{sec:resultados-experimentales}

Una vez elegido MS COCO 2017 dataset como conjunto de datos de referencia para usar con YOLOv4, se va a ver los resultados obtenidos en distintos algoritmos.

\subsection{Resultados en detección de objetos con YOLOv4}
\label{subsec:resultados-yolov4-tf}

\textcolor{red}{Poner imágenes de las detecciones en los datasets empleados y explicar con detalle como influye distintos elementos como la distancia, iluminación, color, etc... en la detección de objetos}.

\url{https://old.photojoiner.net/}

\textcolor{red}{También hacer tablas durante un número determinado de frames donde, a modo de ejemplo se calcule las métricas más relevantes}.

\subsection{Resultados en tracking con DeepSORT}
\label{subsec:resultados-deepsort}

\textcolor{red}{Poner imágenes de las detecciones en los datasets empleados y explicar con detalle los problemas con los que nos podemos encontrar a la hora de hacer un seguimiento o \textit{tracking} de objetos y personas, se pierde el rastreo y se vuelve a asociar una nueva ID a ese individuo, también explicar que debido al \textit{threshold} que pongamos se pueden perder objetos, pero tampoco es conveniente bajarlo a más de X porque entonces se confunden objetos o se trackean varias veces}.

\subsection{Resultados en algoritmo de detección de objetos abandonados}
\label{subsec:resultados-abandon-algorithm}

\textcolor{red}{Poner imágenes de detecciones de objetos abandonados y explicar de manera visual como es posible que hayan problemas cuando se pierde el \textit{tracking} de un objeto o persona y se tiene que reasignar como nuevo ID y posteriormente tener que volver a hacer una asociación de persona-objeto y evaluar si se abandona o no dicho objeto}.

\textcolor{red}{Aquí meter un subsubapartado con métricas relevantes en la detección de objetos abandonados}.

\section{Conclusiones}
\label{sec:conclu-resultados}