
%%%%%%%%%%%%%%%%%%%%
% PAQUETES BÁSICOS %
%%%%%%%%%%%%%%%%%%%%

\usepackage[utf8]{inputenc}                         %%% Para poder escribir con acentos y ñ. en utf8
\usepackage[T1]{fontenc}
\usepackage[spanish]{babel}                         %%% Traduce apartados a español
\usepackage{graphicx}                               %%% Permite incluir imágenes
\usepackage{tocbibind}
\usepackage{amsmath}
\usepackage{amsfonts}
\usepackage{emptypage}                              %%% Elimina encabezados y pies de pagina en paginas vacías
\usepackage{fancyhdr}
\usepackage{geometry}
\usepackage{eurosym}
\usepackage[titletoc]{appendix}                     %%% Incluir anexos
\usepackage{listings}                               %%% Para introducir código
\usepackage[table,xcdraw]{xcolor}
\usepackage{pdfpages}                               %%% Para poder incluir paginas de pdf
\usepackage{url}                                    %%% Para introducir url's con formato: \url{http://www.google.es}
\usepackage{amssymb}                                %%% Para poder usar >= o <=
\usepackage[vlined,algochapter]{algorithm2e}        %%% Para poder escribir algoritmos
\usepackage{bbm}                                    %%% Para escribir letras y números góticas en ecuaciones


%%%%%%%%%%%%%%%%%%%%
% FIGURAS Y TABLAS %
%%%%%%%%%%%%%%%%%%%%

% Nombres de tablas y figuras en negrita y pequeño
\usepackage{caption}
\captionsetup{labelfont={bf,small},textfont=small}
% Para poner dos figuras
\usepackage{subcaption}


%%%%%%%%%
% TODOs %
%%%%%%%%%

% Para poder meter TODOs en el pdf
\usepackage[colorinlistoftodos, spanish, textsize=small]{todonotes}
\newcommand{\pendiente}[1]{\todo[color=green!40]{#1}} % De color verde para cosas PENDIENTES \aviso{...texto...}
\newcommand{\aviso}[1]{\todo[color=red!40]{#1}} % De color rojo para cosas IMPORTANTES \pendiente{...texto...}


%%%%%%%%%%%
% COLORES %
%%%%%%%%%%%

% Configuracion colores en citas, urls, etc...
\usepackage[
bookmarks=true,
bookmarksnumbered=true,
hypertexnames=false,
breaklinks=true,
linktoc=all,
colorlinks=true,
citecolor=green,  %%% Color cuando se cita algo de la bibliografía
urlcolor=blue,    %%% Color cuando se escribe una url
pdfborder={0 0 112.0},
hyperfootnotes=false,
]{hyperref}

% Colores índice general, índice de figuras e índice de tablas
\newcommand{\mytoclinkcolor}{black} % Color índice de contenidos
\newcommand{\myloflinkcolor}{black} % Color índice de figuras
\newcommand{\mylotlinkcolor}{black} % Color índice de tablas
\newcommand{\mylollinkcolor}{black} % Color índice de código fuente
\newcommand{\mylinkcolor}{blue} % Resto de colores: referencias a acrónimos, capítulos, secciones, subsecciones, tablas, figuras, ecuaciones...


%%%%%%%%%%%%%%%%%%%%%%%
% MÁRGENES Y ESPACIOS %
%%%%%%%%%%%%%%%%%%%%%%%

% Ajustar tamaño de los márgenes
\geometry{verbose,a4paper,tmargin=2.5cm,bmargin=2.5cm,lmargin=2.5cm,rmargin=2.5cm,marginpar=2cm}

% Incluir un pequeño espacio entre parrafos
\setlength{\parskip}{1ex plus 0.5ex minus 0.2ex}


%%%%%%%%%%%%%%%%%%%%%
% ESTILO ENCABEZADO %
%%%%%%%%%%%%%%%%%%%%%

% Configuración del estilo del encabezado UAH
\pagestyle{fancy}

\fancyhf{}

\fancyhead[LE,RO]{\bfseries\thepage}
\fancyhead[LO]{\bfseries\rightmark}
\fancyhead[RE]{\bfseries\leftmark}

\makeatletter
\renewcommand{\chaptermark}[1]{\markboth{\@chapapp \ \thechapter . \ #1}{}}
\renewcommand{\sectionmark}[1]{\markright{\thesection \ \ #1}}
\makeatother

\renewcommand{\headrulewidth}{0.5pt}
\renewcommand{\footrulewidth}{0pt}
\addtolength{\headheight}{3.5pt}
\fancypagestyle{plain}{\fancyhead{}\renewcommand{\headrulewidth}{0pt}}
\fancypagestyle{myplain}
{
  \fancyhf{}
  \renewcommand\headrulewidth{0pt}
  \renewcommand\footrulewidth{0pt}
  \fancyfoot[C]{\thepage}
}


%%%%%%%%%%%%%%%%%%%%%%%%
% NIVELES DE SECCIONES %
%%%%%%%%%%%%%%%%%%%%%%%%

% Para tener hasta cuatro niveles: section, subsection, subsubsection, paragraph
\setcounter{tocdepth}{4}
\setcounter{secnumdepth}{4}


%%%%%%%%%%%%%
% ACRÓNIMOS %
%%%%%%%%%%%%%

% Importante cargar este paquete después de \usepackage[hyperref], si no no funciona los hipervínculos
\usepackage[acronym,nonumberlist,toc]{glossaries}
\setlength{\glsdescwidth}{0.7\linewidth} % Ancho de la columna de descripción de los acrónimos


%%%%%%%%%%%%%%%%%
% FRASE CÉLEBRE %
%%%%%%%%%%%%%%%%%

\newenvironment{FraseCelebre}
{\begin{list}{}{
      \setlength{\leftmargin}{0.5\textwidth}
      \setlength{\parsep}{0cm}
      \addtolength{\topsep}{0.5cm}
    }
  }
  {\unskip \end{list}}

\newenvironment{Frase}
{\item \begin{flushright}\small\em}
  {\end{flushright}}

\newenvironment{Fuente}
{\item \begin{flushright}\small}
  {\end{flushright}}
