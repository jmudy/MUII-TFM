
%%%%%%%%%%%%%%%%%%%%
% PAQUETES BÁSICOS %
%%%%%%%%%%%%%%%%%%%%

\usepackage[utf8]{inputenc} % Para poder escribir con acentos y ñ. en utf8
\usepackage[T1]{fontenc}
\usepackage[spanish]{babel}     %%% Traduce apartados a español
\usepackage{graphicx}           %%% Permite incluir imagenes
\usepackage{tocbibind}
\usepackage{amsmath}
\usepackage{amsfonts}
\usepackage{emptypage}          %%% Elimina encabezados y pies de pagina en paginas vacias
\usepackage{fancyhdr}
\usepackage{geometry}
\usepackage{eurosym}
\usepackage[titletoc]{appendix} %%% Incluir anexos
\usepackage{listings}           %%% Para introducir código
\usepackage[table,xcdraw]{xcolor}
\usepackage{pdfpages}           %%% Para poder incluir paginas de pdf
\usepackage{url}                %%% Para introducir url's con formato: \url{http://www.google.es}

%%%%%%%%%%%%%%%%%%%%
% FIGURAS Y TABLAS %
%%%%%%%%%%%%%%%%%%%%

% Nombres de tablas y figuras en negrita y pequeño
\usepackage{caption}
\captionsetup{labelfont={bf,small},textfont=small}
% Para poner una imagen al lado de otra
\usepackage{subcaption}

%%%%%%%%%
% TODOs %
%%%%%%%%%

% Para poder meter TODOs en el pdf
\usepackage[colorinlistoftodos, spanish, textsize=small]{todonotes}
\newcommand{\pendiente}[1]{\todo[color=green!40]{#1}}
\newcommand{\aviso}[1]{\todo[color=red!40]{#1}}

%%%%%%%%%%%
% COLORES %
%%%%%%%%%%%

% Configuracion colores en citas, urls, etc...
\usepackage[
bookmarks=true,
bookmarksnumbered=true,
hypertexnames=false,
breaklinks=true,
linktoc=all,
colorlinks=true,
linkcolor=blue,   %%% Por defecto es de color rojo los índices 
citecolor=green,  %%% Por defecto las citas salen de color verde
urlcolor=blue,    %%% Por defecto las url salen de color magenta en la bibliografía
pdfborder={0 0 112.0},
hyperfootnotes=false,
]{hyperref}

% Colores índice general, índice de figuras e índice de tablas
\newcommand{\mytoclinkcolor}{black}
\newcommand{\myloflinkcolor}{black}
\newcommand{\mylotlinkcolor}{black}

% Otros colores de links del documento
\newcommand{\mylinkcolor}{blue}

%%%%%%%%%%%%%%%%%%%%%%%
% MÁRGENES Y ESPACIOS %
%%%%%%%%%%%%%%%%%%%%%%%

% Ajustar tamaño de los margenes

\geometry{verbose,a4paper,tmargin=2.5cm,bmargin=2.5cm,lmargin=2.5cm,rmargin=2.5cm,marginpar=2cm}

% Incluir un pequeño espacio entre parrafos
\setlength{\parskip}{1ex plus 0.5ex minus 0.2ex}

%%%%%%%%%%%%%%%%%%%%%
% ESTILO ENCABEZADO %
%%%%%%%%%%%%%%%%%%%%%

% Configuración del estilo del encabezado
\pagestyle{fancy}

\fancyhf{}

\fancyhead[LE,RO]{\bfseries\thepage}
\fancyhead[LO]{\bfseries\rightmark}
\fancyhead[RE]{\bfseries\leftmark}

\makeatletter
\renewcommand{\chaptermark}[1]{\markboth{\@chapapp \ \thechapter . \ #1}{}}
\renewcommand{\sectionmark}[1]{\markright{\thesection \ \ #1}}
\makeatother

\renewcommand{\headrulewidth}{0.5pt}
\renewcommand{\footrulewidth}{0pt}
\addtolength{\headheight}{3.5pt}
\fancypagestyle{plain}{\fancyhead{}\renewcommand{\headrulewidth}{0pt}}
\fancypagestyle{myplain}
{
  \fancyhf{}
  \renewcommand\headrulewidth{0pt}
  \renewcommand\footrulewidth{0pt}
  \fancyfoot[C]{\thepage}
}

% Para tener hasta cuatro niveles: section, subsection, subsubsection, paragraph
\setcounter{tocdepth}{4}
\setcounter{secnumdepth}{4}