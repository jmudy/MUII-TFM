
\usepackage[utf8]{inputenc}     %%% Para poder escribir con acentos y ñ. en UTF-8
\usepackage[T1]{fontenc}
\usepackage[spanish]{babel}     %%% Traduce apartados a español
\usepackage{graphicx}           %%% Permite incluir imagenes
\usepackage{tocbibind}
\usepackage{lipsum}
\usepackage{emptypage}          %%% Elimina encabezados y pies de pagina en paginas vacias
\usepackage{fancyhdr}
\usepackage{geometry}
\usepackage{eurosym}
\usepackage[titletoc]{appendix} %%% Incluir anexos
\usepackage{listings}           %%% Para introducir código
\usepackage{xcolor}
\usepackage{pdfpages}           %%% Para poder incluir paginas de pdf

% Para poder meter TODOs en el pdf
\usepackage[colorinlistoftodos, spanish, textsize=small]{todonotes}
\newcommand{\tribunal}[1]{\todo[color=green!40]{#1}}


% Estilo del codigo escrito en el documento
\lstset{language=Python,backgroundcolor=\color{lightgray},emph={python},emphstyle=\textbf}


% Configuracion colores, hipervinculos y marcadores
\usepackage[
bookmarks=true,
bookmarksnumbered=true,
hypertexnames=false,
breaklinks=true,
linktoc=all,
colorlinks=true,
linkcolor=blue,    
citecolor=green,
urlcolor=blue,
pdfborder={0 0 112.0},
hyperfootnotes=false,
]{hyperref}       

% Ajustar tamaño de los margenes

\geometry{verbose,a4paper,tmargin=2.5cm,bmargin=2.5cm,lmargin=2.5cm,rmargin=2.5cm,marginpar=2cm}

% Incluir un pequeño espacio entre parrafos
\setlength{\parskip}{1ex plus 0.5ex minus 0.2ex}

% Configuración del estilo del encabezado

\pagestyle{fancy}

\fancyhf{}

\fancyhead[LE,RO]{\bfseries\thepage}
\fancyhead[LO]{\bfseries\rightmark}
\fancyhead[RE]{\bfseries\leftmark}

\makeatletter
\renewcommand{\chaptermark}[1]{\markboth{\@chapapp \ \thechapter . \ #1}{}}
\renewcommand{\sectionmark}[1]{\markright{\thesection \ \ #1}}
\makeatother

\renewcommand{\headrulewidth}{0.5pt}
\renewcommand{\footrulewidth}{0pt}
\addtolength{\headheight}{3.5pt}
\fancypagestyle{plain}{\fancyhead{}\renewcommand{\headrulewidth}{0pt}}
\fancypagestyle{myplain}
{
  \fancyhf{}
  \renewcommand\headrulewidth{0pt}
  \renewcommand\footrulewidth{0pt}
  \fancyfoot[C]{\thepage}
}

\lstdefinestyle{console}
{
  basicstyle=\scriptsize\bf\ttfamily,
  backgroundcolor=\color{gray75},
}