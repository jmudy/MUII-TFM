
\chapter*{Resumen extendido}
\label{cha:resumen-extendido}

\addcontentsline{toc}{chapter}{Resumen extendido}

La detección de objetos abandonados se trata de una de las aplicaciones más importantes dentro de los sistemas de detección por videovigilancia en los últimos años \cite{DBLP:journals/spm/PlataniotisR05}. La demanda de detección de objetos abandonados está al alza y se precisa disponer de aplicaciones capaces de detectar y evaluar conductas en tiempo real y con márgenes de error reducidos. Este trabajo pretende cubrir una de las etapas de desarrollo en el rastreo, la asociación entre persona y objeto, con la finalidad de poder identificar al propietario y determinar si ha abandonado el objeto o no.

En este trabajo se realizado un estudio exhaustivo del Estado del Arte actual en estrategias para abordar la problemática de la detección de objetos abandonados mediante el uso de aplicaciones de videovigilancia.

Se ha implementado y evaluado YOLOv4 \cite{bochkovskiy2020yolov4}, un algoritmo de detección de objetos que hace uso de una única red neuronal convolucional para detectar objetos a partir de imágenes. Esta red neuronal, que está previamente entrenada con el dataset de MS COCO \cite{lin2015microsoft}, ha vuelto a ser entrenada con el dataset Open Images Dataset v4 \cite{Kuznetsova_2020} con el objetivo de que solo detecte ciertos objetos de interés: personas, mochilas, bolsas de mano y maletas. Tras el entrenamiento se han calculado las métricas de calidad más utilizadas en la evaluación de algoritmos de detección de objetos para determinar si se superan las del modelo pre-entrenado del dataset de MS COCO.

El dataset de referencia con el que se obtuvieron mejores métricas en YOLOv4 fue MS COCO. Posteriormente se han realizado evaluaciones sobre los datasets de detección de objetos abandonados PETS2007 \cite{pets2007-dataset}, AVSSAB2007 \cite{AVSSAB2007-dataset}, GBA2018 \cite{gba-dataset} y ABODA \cite{aboda-dataset}. 

En base a YOLOv4 se ha realizado un estudio en el Estado del Arte de los algoritmos de seguimiento más actuales con la finalidad de que asignar una identidad a cada detección. Se ha implementado el algoritmo Deep SORT \cite{Wojke2017simple} junto a YOLOv4. Deep SORT es un algoritmo predecesor de SORT \cite{Bewley_2016}, que realiza un seguimiento basado en la detección, realizando los procesos de predicción y actualización con filtros de Kalman. Empleando este algoritmo de seguimiento se ha podido rastrear el movimiento de las personas y los objetos asignándoles una identidad única. Del mismo modo que en el algoritmo de detección, se ha evaluado su funcionamiento sobre los datasets más relevantes.

Posteriormente se ha diseñado, implementado y evaluado un algoritmo que determine si un objeto ha sido abandonado o no en base a los algoritmos de detección y seguimiento antes nombrados. Para ello, se ha calculado en los 5 primeros segundos del vídeo la distancia existente entre las personas con todos los objetos de interés detectables. Con la distancia mínima que exista entre una persona y un objetos se puede establecer una asociación.

Obtenida la vinculación persona-objeto se puede evaluar el comportamiento calculando la distancia en píxeles a la que se encuentran en los siguientes fotogramas del vídeo, y así determinar si se produce un abandono del objeto. Otra posibilidad es que el objeto se encuentre durante el transcurso de todo el vídeo estático \cite{luna2018} en el mismo punto y sin una persona asociada. En este caso se puede deducir que ese objeto está abandonado sin posibilidad de detectar al propietario.

Con el desarrollo del algoritmo capaz de detectar objetos abandonados se ha evaluado los resultados en distintos escenarios, del mismo modo que con los algoritmos de detección y seguimiento, teniendo como métrica de calidad la tasa de fallos en la determinación de si un objeto ha sido abandonado.
