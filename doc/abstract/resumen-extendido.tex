
\chapter*{Resumen extendido}
\label{cha:resumen-extendido}

\addcontentsline{toc}{chapter}{Resumen extendido}

La detección de objetos abandonados se trata de una de las aplicaciones más importantes dentro de los sistemas de detección por videovigilancia en los últimos años \cite{DBLP:journals/spm/PlataniotisR05}. La demanda de detección de objetos abandonados esta al alza y se precisa disponer de aplicaciones capaces de detectar y evaluar conductas en tiempo real y con márgenes de error reducidos. Una de las fases de desarrollo en la detección se trata de la asociación entre la persona y objeto que ha sido abandonado, con la finalidad de poder identificar el propietario del mismo y determinar si ese objeto ha sido abandonado o robado.

En este trabajo se realizado un exhaustivo estudio del Estado del Arte actual de estrategias para abordar la problemática de la detección de objetos abandonados mediante e uso de aplicaciones de videovigilancia.

Se ha implementado y evaluado YOLOv4 \cite{bochkovskiy2020yolov4}, un algoritmo de detección de objetos que hace uso de una sola red neuronal convolucional para detectar objetos a partir de imágenes. Esta red neuronal, que está previamente entrenada con el dataset de MS COCO 2017 \cite{lin2015microsoft}, ha vuelto a ser reentrenada con el dataset Open Images Dataset v4 \cite{Kuznetsova_2020} para que solo detecte ciertos objetos: personas, mochilas, bolsos, bolsas de mano, maletines y maletas. Tras el entrenamiento se han calculado las métricas de calidad más utilizadas construyendo la matriz de confusión para determinar si con ese dataset de referencia supera al modelo preentrenado con el dataset de MS COCO 2017 \cite{lin2015microsoft}.

El dataset que mejores resultados se obtuvo con YOLOv4 \cite{bochkovskiy2020yolov4} fue MS COCO 2017. Posteriormente se ha realizado una evaluación sobre datasets de detección de objetos abandonados GBA2018 \cite{gba-dataset}, PETS2007 \cite{pets2007-dataset} o AVSSAB2007 \cite{AVSSAB2007-dataset}. 

En base a YOLOv4 se ha realizado un estudio en el Estado del Arte de los algoritmos de seguimiento más actuales con la finalidad de que asignar una identidad a cada detección. Se ha implementado el algoritmo DeepSORT \cite{Wojke2017simple} junto a YOLOv4. DeepSORT  es un algoritmo predecesor de SORT \cite{Bewley_2016}, que realiza un seguimiento basado en la detección, realizando los procesos de predicción y actualización con filtros de Kalman. Empleando este algoritmo de seguimiento se ha podido rastrear el movimiento de las personas y los objetos asignándoles una identidad única. Del mismo modo que en el algoritmo de detección se ha evaluado su funcionamiento sobre los datasets más relevantes.

Posteriormente se ha diseño, implementado y evaluado un algoritmo que determine si un objeto ha sido abandonado o no a partir de los algoritmos de detección y seguimiento antes nombrados. Para ello, se ha calculado en los 5 primeros segundos del vídeo la distancia existente entre las personas con todos los objetos de interés detectables. Con la distancia mínima que exista entre una persona y un objetos se puede establecer una asociación.

Obtenida la vinculación persona-objeto se puede evaluar el comportamiento del mismo calculando la distancia a la que se encuentran en los siguiente \textit{frames} del video y determinar si se produce un abandono del objeto o no. Otra posibilidad es que el objeto se encuentre durante el transcurso de todo el vídeo estático \cite{article} en el mismo punto y sin asignación con otra persona. En este caso se puede deducir que ese objeto ha sido abandonado sin posibilidad de detectar al propietario.

Con el desarrollo del algoritmo capaz de detectar objetos abandonados se ha evaluado los resultados en distintos escenarios del mismo modo que con los algoritmo de detección y seguimiento, teniendo como métrica de calidad la tasa de fallos en la determinar si un objeto ha sido abandonado.
