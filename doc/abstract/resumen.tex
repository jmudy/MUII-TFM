
\chapter*{Resumen}
\label{cha:resumen}

\addcontentsline{toc}{chapter}{Resumen}

Este Trabajo Fin de Máster propone el estudio e implementación de algoritmos de aprendizaje profundo (\textit{Deep Learning}) con la finalidad de detectar objetos abandonados en aplicaciones de videovigilancia.

Se ha realizado un estudio teórico de los algoritmos de detección y seguimiento disponibles en el Estado del Arte. Para la detección de objetos en tiempo real se ha empleado YOLOv4 \cite{bochkovskiy2020yolov4}. Como algoritmo de seguimiento se ha optado por DeepSORT \cite{Wojke2017simple}. Por último, se ha desarrollado un algoritmo que determine si un objeto ha sido abandonado o no. Todos ellos han sido implementados sobre el set de datos de referencia MS COCO 2017 \cite{lin2015microsoft} y evaluados sobre las bases de datos más relevantes en la detección de objetos abandonados como son GBA 2018 \cite{gba-dataset}, PETS 2007 \cite{pets2007-dataset} o AVSS AB 2007 \cite{AVSSAB2007-dataset}.

\textbf{Palabras clave:} Deep Learning, YOLOv4, DeepSORT, videovigilancia, visión por computador.