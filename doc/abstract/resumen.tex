
\chapter*{Resumen}
\label{cha:resumen}

\addcontentsline{toc}{chapter}{Resumen}

Este trabajo aborda el estudio e implementación de algoritmos de aprendizaje profundo (\textit{Deep Learning}) con la finalidad de detectar objetos abandonados en aplicaciones de videovigilancia.

Se ha realizado un estudio teórico de los algoritmos de detección y seguimiento disponibles en el Estado del Arte. Para la detección de objetos en tiempo real se ha empleado YOLOv4 \cite{bochkovskiy2020yolov4}. Como algoritmo de seguimiento se ha optado por Deep SORT \cite{Wojke2017simple}. Por último, se ha desarrollado un algoritmo que determine si un objeto ha sido abandonado o no. Todos ellos han sido implementados sobre el dataset de referencia MS COCO \cite{lin2015microsoft} y evaluados sobre los datasets más relevantes en la detección de objetos abandonados como son GBA2018 \cite{gba-dataset}, PETS2007 \cite{pets2007-dataset}, AVSSAB2007 \cite{AVSSAB2007-dataset} o ABODA \cite{aboda-dataset}.

\textbf{Palabras clave:} Deep Learning, YOLOv4, Deep SORT, videovigilancia, visión por computador.