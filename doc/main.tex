
\documentclass[spanish,openright]{book}

\usepackage[utf8]{inputenc}
\usepackage[english]{babel}
\usepackage{graphicx}
\usepackage[nottoc]{tocbibind} % para que muestre el apartado de referencias en el índice
\usepackage{lipsum} % Para meter texto random
\input{config/python-style.tex}

\makeglossaries % Genera la base de datos de acrónimos
\newacronym{aboda}{ABODA}{Abandoned Objects Dataset}
\newacronym{ap}{AP}{Average precision}
\newacronym{asff}{ASFF}{Adaptively Spatial Feature Fusion}
\newacronym{auc}{AUC}{Area Under the ROC Curve}
\newacronym{avss}{AVSSAB2007}{Advanced Video and Signal based Surveillance Abandoned Baggage}


\newacronym{bow}{BOW}{Bag of Words}


\newacronym{cnn}{CNN}{Redes Neuronales Convolucionales}
\newacronym{coco}{MS COCO}{Microsoft Common Objects in Context (test-dev 2017)}
\newacronym{cpu}{CPU}{Central Processing Unit}
\newacronym{cuda}{CUDA}{Compute Unified Device Architecture}
\newacronym{cudnn}{cuDNN}{CUDA Deep Neural Network}


\newacronym{deepsort}{Deep SORT}{Simple Online and Realtime Tracking with a Deep Association Metric}
\newacronym{dnn}{DNN}{Deep Neural Network}
\newacronym{dssd}{DSSD}{Deconvolutional Single Shot Detector}


\newacronym{eps}{EPS}{Escuela Politécnica Superior}


\newacronym{fairmot}{FairMOT}{On the Fairness of Detection and Re-Identification in Multiple Object Tracking}
\newacronym{fn}{FN}{False Negative}
\newacronym{fp}{FP}{False Positive}
\newacronym{fc}{FC}{Fully Connected}
\newacronym{flops}{FLOPS}{Floating Point Operations Per Second}
\newacronym{fpn}{FPN}{Feature Pyramid Network}
\newacronym{fps}{FPS}{Frames Per Second}


\newacronym{gba2018}{GBA2018}{GEINTRA Behaviour Analysis 2018}
\newacronym{geintra}{GEINTRA}{Grupo de Ingeniería Electrónica Aplicada a Espacios Inteligentes y Transporte}
\newacronym{gpu}{GPU}{Graphics Processing Unit}


\newacronym{hog}{HOG}{Histograma de Gradientes Orientados}


\newacronym{ide}{IDE}{Integrated Development Environment}
\newacronym{iou}{IoU}{Intersection over Union}
\newacronym{ipm}{IPM}{Inverse Perspective Mapping}


\newacronym{map}{mAP}{Mean average precision}
\newacronym{mht}{MHT}{Multiple Hypothesis Tracking}
\newacronym{mlp}{MLP}{Multilayer perceptron}
\newacronym{mot}{MOT}{Multi Object Tracking}


\newacronym{nas}{NAS}{Network Architecture Search}


\newacronym{pan}{PAN}{Path Aggregation Network}
\newacronym{pets}{PETS2007}{Performance Evaluation of Tracking and Surveillance 2007}


\newacronym{oidv4}{OIDv4}{Open Images Dataset v4}


\newacronym{r-cnn}{R-CNN}{Region Based Convolutional Neural Networks}
\newacronym{rgb}{RGB}{Red, Green, Blue}
\newacronym{roi}{ROI}{Region of interest}
\newacronym{rpn}{RPN}{Region Proposal Network}


\newacronym{sfam}{SFAM}{Scale-wise Feature Aggregation Module}
\newacronym{sort}{SORT}{Simple Online and Realtime Tracking}
\newacronym{sot}{SOT}{Single Object Tracking}
\newacronym{spp}{SPP}{Spatial Pyramid Pooling}
\newacronym{ssd}{SSD}{Single Shot MultiBox Detector}
\newacronym{svdd}{SVDD}{Support Vector Data Description}
\newacronym{svm}{SVM}{Support Vector Machines}


\newacronym{tfm}{TFM}{Trabajo Fin de Máster}
\newacronym{tn}{TN}{True Negative}
\newacronym{tp}{TP}{True Positive}


\newacronym{uah}{UAH}{Universidad de Alcalá de Henares}


\newacronym{yolo}{YOLO}{You Only Look Once}
\newacronym{yolov4}{YOLOv4}{You Only Look Once v4} % Archivo que contiene los acrónimos

\begin{document}

% Portada
\includepdf[pages={1-3}]{cover/portada.pdf}

% Numeración romana
\frontmatter

% Dedicatoria + agradecimientos
\thispagestyle{empty}

\begin{flushright}

  \topskip0pt
  \vspace*{\fill}

  \textbf{A mi hermano Carlos\ldots}\\

  \vspace{3cm}

  \tribunal{Falta añadir el tribunal en la página anterior y la fecha de depósito! cambiarlo en el archivo ./Config/myconfig.tex del repositorio plantilla}\emph{``Empieza haciendo lo necesario, luego haz lo posible y de
    pronto empezarás a hacer lo imposible.''}\\ Francisco de Asís

\end{flushright}  

\vspace{4cm}
\vspace*{\fill}

\chapter*{Agradecimientos}
\label{cha:agradecimientos}
\markboth{Agradecimientos}{Agradecimientos}

Este trabajo es el fruto de muchas horas de trabajo, tanto de los
autores últimos de los ficheros de la distribución como de todos los que
en mayor o menor medida han participado en él a lo largo de su proceso
de gestación.

Mención especial merece Manuel Ocaña, el autor de la primera versión de
las plantillas de proyectos fin de carrera y tesis doctorales usadas en
el Departamento de Electrónica de la Universidad de Alcalá, con
contribuciones de Jesús Nuevo, Pedro Revenga, Fernando Herránz y Noelia
Hernández.

En la versión actual, la mayor parte de las definiciones de estilos de
partida proceden de la tesis doctoral de Roberto Barra-Chicote, con lo
que gracias muy especiales para él.

También damos las gracias a \dots que nos
han proporcionado secciones completas y ejemplos puntuales de sus
proyectos fin de carrera.

Finalmente, hay incontables contribuyentes a esta plantilla, la mayoría
encontrados gracias a la magia del buscador de Google. Hemos intentado
referenciar los más importantes en los fuentes de la plantilla, aunque
seguro que hemos omitido alguno. Desde aquí les damos las gracias a
todos ellos por compartir su saber con el mundo.

% Resumen/abstract + resumen extendido

\chapter*{Resumen}
\label{cha:resumen}

\addcontentsline{toc}{chapter}{Resumen}

Este Trabajo Fin de Máster tiene como objeto el estudio e implementación de algoritmos empleando redes neuronales convolucionales (\textit{CNN}) con la finalidad de detectar objetos abandonados mediante el uso de aplicaciones de videovigilancia. Estas redes se tratan de algoritmos de aprendizaje supervisado especializados para trabajar con imágenes (poner explicación de DotCSV).

En el presente trabajo se ha realizado un estudio teórico de los distintos algoritmos de detección de objetos sobre determinadas bases de datos así como algoritmos de rastreo o \textit{tracking} disponibles en el Estado del Arte. Para la detección de objetos se ha empleado YOLOv4 \cite{bochkovskiy2020yolov4}. Posteriormente se ha desarrollado un algoritmo de DeepSORT \cite{Wojke2017simple} donde se ha filtrado que rastree únicamente a personas y objetos de interés: mochilas, maletas, bolsos y bolsas de mano. Se ha utilizado COCO \cite{lin2015microsoft} como conjunto de datos ya que se trata de un estándar de referencia muy utilizado en la evaluación del rendimiento de modelos de visión por computador.

Por último se ha implementado y evaluado un algoritmo que determine si un objeto ha sido abandonado o no. Para ello se deberán de tener en cuenta diferentes métricas como el tiempo que el objeto se encuentra estático en un determinado punto o la distancia a la que se encuentra respecto a la persona que lo portaba.

\textbf{Palabras clave:} Redes neuronales convolucionales, YOLOv4, DeepSORT, videovigilancia, visión por computador.

\chapter*{Abstract}
\label{cha:abstract}

This document has been generated with a template for Bsc and Msc Thesis
(trabajos fin de carrera, fin de máster, fin de grado) and PhD. Thesis,
specially thought for its use in Universidad de Alcalá, although it
should be easily extended and adapted for other use cases. In its
content we include general instructions of use, and some example of
elements than can be useful. If you have problems, suggestions or
comments on the template, please forward them to \ldots.

\textbf{Keywords:} bla, bla, bla.

\chapter*{Resumen extendido}
\label{cha:resumen-extendido}

\addcontentsline{toc}{chapter}{Resumen extendido}

La detección de objetos abandonados se trata de una de las aplicaciones más importantes dentro de los sistemas de detección por videovigilancia en los últimos años \cite{DBLP:journals/spm/PlataniotisR05}. La demanda de detección de objetos abandonados esta al alza y se precisa disponer de aplicaciones capaces de detectar y evaluar conductas en tiempo real y con márgenes de error reducidos. Este trabajo pretende cubrir una de las etapas de desarrollo en la detección, la asociación entre persona y objeto, con la finalidad de poder identificar al propietario y determinar si el objeto ha sido abandonado o no.

En este trabajo se realizado un estudio exhaustivo del Estado del Arte actual en estrategias para abordar la problemática de la detección de objetos abandonados mediante el uso de aplicaciones de videovigilancia.

Se ha implementado y evaluado YOLOv4 \cite{bochkovskiy2020yolov4}, un algoritmo de detección de objetos que hace uso de una única red neuronal convolucional para detectar objetos a partir de imágenes. Esta red neuronal, que está previamente entrenada con el dataset de MS COCO \cite{lin2015microsoft}, ha vuelto a ser entrenada con el dataset Open Images Dataset v4 \cite{Kuznetsova_2020} con el objetivo de que solo detecte ciertos objetos de interés: personas, mochilas, bolsos, bolsas de mano, maletines y maletas. Tras el entrenamiento se han calculado las métricas de calidad más utilizadas en la evaluación de algoritmos de detección de objetos para determinar si se superan las del modelo preentrenado del dataset de MS COCO.

El dataset de referencia con el que se obtuvieron mejores métricas en YOLOv4 fue MS COCO. Posteriormente se han realizado evaluaciones sobre los datasets de detección de objetos abandonados PETS2007 \cite{pets2007-dataset}, AVSSAB2007 \cite{AVSSAB2007-dataset}, GBA2018 \cite{gba-dataset} y ABODA \cite{aboda-dataset}. 

En base a YOLOv4 se ha realizado un estudio en el Estado del Arte de los algoritmos de seguimiento más actuales con la finalidad de que asignar una identidad a cada detección. Se ha implementado el algoritmo Deep SORT \cite{Wojke2017simple} junto a YOLOv4. Deep SORT  es un algoritmo predecesor de SORT \cite{Bewley_2016}, que realiza un seguimiento basado en la detección, realizando los procesos de predicción y actualización con filtros de Kalman. Empleando este algoritmo de seguimiento se ha podido rastrear el movimiento de las personas y los objetos asignándoles una identidad única. Del mismo modo que en el algoritmo de detección, se ha evaluado su funcionamiento sobre los datasets más relevantes.

Posteriormente se ha diseño, implementado y evaluado un algoritmo que determine si un objeto ha sido abandonado o no en base a los algoritmos de detección y seguimiento antes nombrados. Para ello, se ha calculado en los 5 primeros segundos del vídeo la distancia existente entre las personas con todos los objetos de interés detectables. Con la distancia mínima que exista entre una persona y un objetos se puede establecer una asociación.

Obtenida la vinculación persona-objeto se puede evaluar el comportamiento calculando la distancia a la que se encuentran en los siguientes fotogramas del vídeo, y así determinar si se produce un abandono del objeto. Otra posibilidad es que el objeto se encuentre durante el transcurso de todo el vídeo estático \cite{luna2018} en el mismo punto y sin asignación con otra persona. En este caso se puede deducir que ese objeto está abandonado sin posibilidad de detectar al propietario.

Con el desarrollo del algoritmo capaz de detectar objetos abandonados se ha evaluado los resultados en distintos escenarios, del mismo modo que con los algoritmo de detección y seguimiento, teniendo como métrica de calidad la tasa de fallos en la determinar si un objeto ha sido abandonado.


% Índices
\hypersetup{linkcolor=\mytoclinkcolor}
\tableofcontents

\hypersetup{linkcolor=\myloflinkcolor}
\listoffigures

\hypersetup{linkcolor=\mylotlinkcolor}
% Para que salga "Índice de tablas" en vez de "Índice de cuadros"
\renewcommand{\listtablename}{Índice de tablas} 
\renewcommand{\tablename}{Tabla}
\listoftables

\hypersetup{linkcolor=\mylollinkcolor}
% Para que salga "Índice de códigos" en vez de "Listings"
\renewcommand{\lstlistlistingname}{Índice de códigos}
\renewcommand{\lstlistingname}{Código}
\lstlistoflistings

% Resto de colores
\hypersetup{linkcolor=\mylinkcolor}

% Numeración normal
\mainmatter

% Capítulos

\chapter{Introducción}
\label{cha:introduccion}

En los últimos años \ldots

o

En este capítulo de introducción se quiere \ldots

\section{Motivación}
\label{sec:motivacion}

Introducción del proyecto haciendo alusiones a trabajos previos para poner en contexto, hablar sobre las técnicas empleadas y desarrollarlas brevemente para poner en contexto. Sin extenderse en en exceso ya que se desarrollará más en el apartado del Estado del Arte del capítulo 2 \ldots

\section{Objetivos}
\label{sec:objetivos}

El objetivo que se quiere llevar a cabo es el desarrollo de una estrategia de detección de objetos abandonados mediante el uso de aplicaciones de videovigilancia. En concreto se va a estudiar cuando se ha abandonado los siguientes tipos de objetos: mochilas, bolsos, maletines, bolsas de mano o maletas. Los espacios donde se va a evaluar la eficacia del sistema de detección desarrollado será tanto interiores como exteriores: estaciones de metro, centros comerciales, colegios/universidades o cualquier tipo de infraestructura que disponga de una o varias cámaras de videovigilancia. 

Los pasos para abordar este problema son los siguientes.

\begin{itemize}
    \item \textbf{Estudio del Estado de Arte actual}. Búsqueda y estudio de artículos referentes a la identificación de objetos abandonados en aplicaciones de videovigilancia dentro del Estado del Arte actual para tener un punto de partida. Por otro lado se deberá de buscar las bases de datos más relevantes en la evaluación de detección de objetos abandonados.
    \item \textbf{Evaluación de algoritmos algoritmos de detección de objetos más relevantes}. Se estudiará y comparará los algoritmos de detección de objetos actuales y se argumentará el motivo de la elección de uno concreto. Una vez seleccionado el algoritmo de detección se deberá de evaluar si trabajar sobre un conjunto de datos conocido o si por el contrario es interesante el entrenamiento de una red neuronal personalizada en la que se detecten solamente los objetos de interés. La elección del conjunto de datos de referencia para la evaluación del algoritmo de detección se decidirá teniendo en cuenta las principales métricas de clasificación de \textit{Machine Learning} así como la matriz de confusión. Teniendo un conjunto de datos de referencia se ejecutará el algoritmo en las bases de datos más utilizadas para evaluar las predicciones.
    \item \textbf{Evaluación de algoritmos de seguimiento o \textit{tracking} de objetos más relevantes}. En base al modelo del algoritmo de detección de objetos seleccionado se estudiará y evaluará los algoritmos de seguimiento actuales. El objetivo de este punto es que en la detección de objetos y personas, cada elemento tenga una identidad propia a lo largo del tiempo, o lo que es lo mismo, a lo largo de los frames de un video. De tal manera que, cuando se implemente el algoritmo de detección de objetos abandonados sea más sencillo la asociación de persona-objeto. De igual manera que en el algoritmo de detección, también se ejecutará el algoritmo en las bases de datos más relevantes en detección de objetos abandonados para evaluar si \textit{tracking} sobre personas y objetos de interés a lo largo de un video.
    \item \textbf{Implementación y evaluación de un algoritmo de detección de objetos abandonados}. Se desarrollará un algoritmo capaz de determinar si un objeto ha sido abandonado o no. Existen tres posibles escenarios. El primero es que el objeto se encuentre móvil durante toda la ejecución del video y no se pueda asociar a ninguna persona como propietario. La segunda es que a una persona a la que se le ha asociado un objeto se alejen más de una cierta distancia a lo largo de un número determinado frames. La tercera es que a una persona a la que se le ha asociado un objeto desaparezca y se este detectando el únicamente el objeto durante un número determinado de frames. Para estos dos últimos casos se deberá de establecer una asociación persona-objeto y estudiar su comportamiento a lo largo del video.
\end{itemize}

\section{Estructura de la memoria}
\label{sec:estructura-memoria}
En este apartado se resume brevemente como se encuentra organizados los contenidos que componen el presente Trabajo Fin de Máster.

\begin{itemize}
    \item \textbf{Capítulo \ref{cha:introduccion}: Introducción.} Se expondrá la motivación que ha impulsado la realización de este Trabajo Fin de Máster. Se citará brevemente trabajos previos que han servido de esqueleto del proyecto. Por otro lado se argumentarán los objetivos que se pretenden alcanzar.
    \item \textbf{Capítulo \ref{cha:estudio-teorico}: Estudio teórico.} Se realizará un estudio exhaustivo del Estado del Arte en lo referente a algoritmos de detección de objetos abandonados en aplicaciones de videovigilancia. Se evaluará y comparará los algoritmos de detección de objetos y algoritmos de seguimiento más relevantes y se desarrollarán en más profundidad los seleccionados argumentando los motivos. 
    \item \textbf{Capítulo \ref{cha:desarrollo}: Desarrollo algoritmo de detección de objetos abandonados.} Se desarrollará un algoritmo de detección de objetos abandonados en el que se tendrá que tener en cuenta si el objeto tiene o no propietario y en el caso de que lo tenga, crear una asociación persona-objeto para evaluar cuando se produce el abandono del objeto.
    \item \textbf{Capítulo \ref{cha:resultados}: Resultados.} Se expondrán los resultados obtenidos tras el desarrollo del proyecto. Se describirá las bases de datos utilizadas para la evaluación de los algoritmos así como las métricas de calidad más relevantes en cada algoritmo.
    \item \textbf{Capítulo \ref{cha:concl-lineas-futuras}: Conclusiones y líneas futuras.} Se expondrán las conclusiones que se han llegado al finalizar este proyecto. Se explicarán las ventajas y limitaciones que presenta la idea propuesta para su desarrollo. Por otro lado se argumentarán posibles vías de desarrollo derivados de este proyecto, así como nuevos proyectos donde se puedan emplear el mismo algoritmo de detección y seguimiento y solamente se tenga que programar un algoritmo que realice una función concreta.
    \item \textbf{Bibliografía.} Se incluye cada uno de los artículos, repositorios, conjuntos de datos y toda clase de material consultado para la elaboración de este Trabajo Fin de Máster indicando la fecha de último acceso. 
    \item \textbf{Apéndice \ref{cha:pliego-de-condiciones}.} Se hace referencia al pliego de condiciones donde se tendrán en cuenta las especificaciones \textit{hardware} y \textit{software} que se han empleado en el desarrollo de este proyecto.
    \item \textbf{Apéndice \ref{cha:presupuesto}.} Se muestra el presupuesto donde se incluye los costes materiales \textit{hardware} y \textit{software} y el coste de la mano de obra en función a la duración estimada del proyecto.
    \item \textbf{Apéndice \ref{cha:manual-usuario-instalacion}.} Se detallan cada unos de los pasos necesarios para instalar todas las dependencias necesarias para el funcionamiento de los algoritmos. Una vez instalado todo el \textit{software} y librerías se puede consultar el manual de usuario donde se indica como poner en funcionamiento cualquiera de los algoritmos desarrollados en este proyecto.
\end{itemize}






\chapter{Estudio teórico}
\label{cha:estudio-teorico}

Aquí se explicará el estado de arte actual \ldots

\begin{figure}[!h]
\centering
\includegraphics[width=0.8\textwidth]{img/rnn.png}
\caption{\label{fig:rnn}Red neuronal \ldots}
\end{figure}
La figura \ref{fig:rnn} muestra \ldots

\section{Introducción}
\label{sec:intro-sota}

\section{Estado del arte}
\label{sec:sota}

\section{Técnicas utilizadas}
\label{sec:tecnicas-utilizadas}

\subsection{YOLOv4 + Tensorflow 2}
\label{sec:tecnicas-utilizadas1}

\subsection{DeepSort}
\label{sec:tecnicas-utilizadas2}

\section{Conclusiones}
\label{sec:conclu-sota}

\chapter{Algoritmos de detección y seguimiento de objetos}
\label{cha:algorithms}

\section{Introducción}
\label{sec:intro-algorithms}

\section{Algoritmos de detección de objetos}
\label{sec:tecnicas-utilizadas-detection}

\subsection{Fast R-CNN}
\label{subsec:fast-rcnn}

\subsection{Faster R-CNN}
\label{subsec:faster-rcnn}

\subsection{SSD: Single Shot MultiBox Detector}
\label{subsec:ssd}

\subsection{EfficientDet}
\label{subsec:efficientdet}

\subsection{YOLO: You Only Look Once}
\label{subsec:yolo}

Presentar YOLO y hacer énfasis en YOLOv4

\section{Algoritmos de seguimiento de objetos}
\label{sec:tecnicas-utilizadas-tracking}

\subsection{Filtro de Kalman}
\label{subsec:kalman-filter}

En azul y naranja las detecciones, deepsort cuadros delimitadores blancos.

Explicar filtro de Kalman

\textcolor{red}{Meter imágenes de ejemplo de se visualice bien el cuadro delimitador de detección y el de seguimiento}

\subsection{Algoritmo húngaro}
\label{subsec:hungarian-algorithm}

Explicar algoritmo húngaro.

\subsection{DeepSORT}
\label{subsec:deepsort}

Explicar que se trata de un algoritmo que combina El filtro de Kalman con el algoritmo húngaro.

\begin{figure}[ht]
\centering
\includegraphics[width=1\textwidth]{img/chapters/algoritmos/deepsort-example.png}
\caption{\label{fig:tracking-example}Funcionamiento de Deepsort}
\end{figure}

\section{Conclusiones}
\label{sec:conclu-algoritmos}

\textcolor{red}{Aquí explicar con que detector de objetos el cual combinaré con DeepSORT para el desarrollo del algoritmo de detección de objetos abandonados.}


\chapter{Desarrollo algoritmo de detección de objetos abandonados}
\label{cha:desarrollo-object-detection}

\section{Introducción}
\label{sec:intro-algoritmo-abandono}

\section{Desarrollo}
\label{sec:algoritmo-abandono}

\textcolor{red}{Dibujar el esquema que voy a seguir para determinar cuando un objeto ha sido abandonado.}
\url{https://app.diagrams.net/}

\section{Conclusiones}
\label{sec:conclu-object-detection}

\chapter{Resultados}
\label{cha:resultados}

Aquí se mostrarán los resultados del proyecto \ldots

Como dijo \cite{einstein} \ldots

\section{Introducción}
\label{sec:intro-resultados}

\section{Entorno experimental}
\label{sec:desarrollo-resultados}

\subsection{Bases de datos utilizadas}
\label{subsec:bases-datos}

\subsection{Métricas de calidad}
\label{subsec:metricas-calidad}

\subsection{Estrategia y metodología de experimentación}
\label{subsec:estrategia-metodologia}

\section{Resultados experimentales}
\label{sec:resultados-experimentales}

\section{Conclusiones}
\label{sec:conclu-resultados}

\chapter{Conclusiones y líneas futuras}
\label{cha:concl-lineas-futuras}

En esta sección se hablará sobre futuros proyectos derivados de éste \ldots

\section{Conclusiones}
\label{sec:conclusiones-finales}

\section{Líneas futuras}
\label{sec:lineas-futuras}

% Biblio
\bibliography{biblio/biblio.bib}
\bibliographystyle{IEEEtran}

% Acrónimos

\printglossary[style=super,type=\acronymtype,title={Lista de Acrónimos}]
\glsaddallunused

% Anexos
\begin{appendices}
\chapter{Pliego de condiciones}
\label{cha:pliego-de-condiciones}

\section{Introducción}
\label{sec:intro-pliego}

\pendiente{Quizás explicación más extensa en los siguientes subapartados}En este anexo se especifican el software y hardware utilizado en el desarollo del proyecto. Por un lado tenemos el equipo A que se utilizará únicamente para la redacción del presente documento con \LaTeX. Por otro lado, la utilización del equipo B tiene como finalidad poder programar los algoritmos con mejor capacidad de computación dadas sus especificaciones.

\section{Características del equipo A}
\label{sec:caracteristicas-equipoa}

\subsection{Especificaciones hardware del equipo A}
\label{subsec:especificaciones-hardware-equipoa}
\begin{itemize}
  \item Procesador: Intel\textregistered\ Core\textsuperscript{TM} i5-3570K @ 3.4Ghz Box
  \item Tarjeta gráfica: NVIDIA\textregistered\ Gigabyte GeForce GTX\textsuperscript{TM} 770 OC 2GB GDDR5
  \item Memoria intalada (RAM): G.Skill Ares DDR3 1600 PC3-12800 8GB 2x4GB CL9
  \item Almacenamiento 1: Samsung 860 EVO Basic SSD 500GB SATA3
  \item Almacenamiento 2: WD Blue 1TB SATA3
\end{itemize}

\subsection{Especificaciones software del equipo A}
\label{subsec:especificaciones-software-equipoa}
\begin{itemize}
  \item Utilización de un sistema operativo de 64 bits, Windows 10 Pro (compilación de SO 19042.804)
  \item Procesador de textos \LaTeX\ con TexMaker 5.0.4 y MiKTeX-TeX 4.0 (MiKTeX 20.12)
  \item Software de control de versiones Git version 2.29.2.windows.1
\end{itemize}

\section{Características del equipo B}
\label{sec:caracteristicas-segun-equipob}

\subsection{Especificaciones hardware del equipo B}
\label{subsec:especificaciones-hardware-equipob}

\begin{itemize}
  \item Procesador: Intel\textregistered\ Core\textsuperscript{TM} i7-6700HQ @ 2.60 GHz
  \item Tarjeta gráfica: NVIDIA\textregistered\ GeForce GTX\textsuperscript{TM} 960M 2GB GDDR5
  \item Memoria intalada (RAM): 20GB
  \item Almacenamiento 1: Disco duro SSD 250GB
  \item Almacenamiento 2: Disco duro HDD 1TB
\end{itemize}

\subsection{Especificaciones software del equipo B}
\label{subsec:especificaciones-software-equipob}
\begin{itemize}
  \item Utilización de un sistema operativo de 64 bits, Ubuntu 18.04.4 LTS
  \item Entorno de trabajo Anaconda 4.9.1
  \item NVIDIA\textregistered\ CUDA\textsuperscript{TM} Toolkit 10.1.243 
  \item cuDNN v7.6.5 (November 5th, 2019), for CUDA\textsuperscript{TM} 10.1
  \item Lenguaje de programación Python\textsuperscript{TM} 3.7.0 empleando Visual Studio Code como editor de código fuente
\end{itemize}

\chapter{Presupuesto}
\label{cha:presupuesto}

\section{Introducción}
\label{sec:intro-presupuesto}

En este apéndice se muestra el presupuesto del proyecto. Para ello, en los siguientes apartados se detallarán el perfil de personal necesario para desarollarlo, la duración estimada de cada una de las etapas que lo compone, de tal forma que se podrá calcular el coste de mano de obra de manera más precisa.

\section{Equipo de trabajo}
\label{sec:equipo-presupuesto}

\pendiente{Ver que ingeniero es el que se dedica a esto}Para la realización del proyecto se va a necesitar un Ingeniero de \textcolor{red}{Telecomunicaciones} que haya trabajado anteriormente con redes neuronales, y en concreto, con redes neuronales convolucionales, tenga gran dominio de la librería OpenCV y en el lenguaje de programación Python.

\section{Timing}
\label{sec:timing-presupuesto}

Las fases necesarias para la realización del desarrollo son las siguientes.

\begin{itemize}
\item formación inicial y revisión del Estado del Arte (1 mes)
  
  \begin{itemize}
    \item Recopilar información sobre el Estado del Arte: algoritmos, sistemas y bases de datos. (0,25 meses)
    \item Comparación entre diferentes estrategias de detección de objetos. (0,25 meses)
    \item Buscar implementaciones disponibles en el alcance del proyecto (0,25 meses)
    \item Estudiar las redes neuronales convolucionales (CNN) para identificar objetos en imágenes (0,25 meses)
  \end{itemize}
  
\item Diseño, implementación y evaluación de algoritmos y librerías en la detección de personas y objetos (0,5 meses)
  
  \begin{itemize}
  \item Diseño, implementación/adaptación de la estrategia seleccionada para esta tarea (0,2 meses)
  \item Refinar el Estado del Arte (0,1 meses)
  \item Evaluación rigurosa de los algoritmos desarrollados en bases de datos relevantes (0,2 meses)
  \end{itemize}
  
\item Diseño, implementación y evaluación de algoritmos y librerías en la seguimiento/rastero de personas y objetos (1,5 meses)
  
  \begin{itemize}
  \item Diseño, implementación/adaptación de la estrategia seleccionada para esta tarea (0,7 meses)
  \item Refinar el Estado del Arte (0,2 meses)
  \item Evaluación rigurosa de los algoritmos desarrollados en bases de datos relevantes (0,6 meses)
  \end{itemize}

\item Diseño, implementación y evaluación de algoritmos y bibliotecas en la tarea de detección de objetos abandonados (1,5 meses):
  
  \begin{itemize}
  \item Diseño, implementación/adaptación de la estrategia seleccionada para esta tarea (0,7 meses)
  \item Refinar el Estado del Arte (0,2 meses)
  \item Evaluación rigurosa de los algoritmos desarrollados en bases de datos relevantes (0,6 meses)
    \end{itemize}
    
\item Evaluación de la integración final de los algoritmos desarrollados en las bases de datos más relevantes. (0,5 meses)

\item Corrección de errores y comienzo de la redacción de la memoria del proyecto (1 mes)

\item Finalización de la redacción de la memoria del proyecto (0,5 meses)
  
\end{itemize}

Cabe destacar que tareas como la de refinar y recopilar información el Estado del Arte, la comparación de diferentes estrategias de detección y rastreo o la evaluación de los algortimos en las distintas bases de datos se repiten a lo largo de los 6,5 meses que dura el desarrollo del proyecto.

\section{Costes}
\label{sec:costes-presupuesto}

\section{Presupuesto total}
\label{sec:presupuesto-total}

Por todo lo anterior, supone una duración de 8 meses naturales con un coste de 12.000 \euro (sin IVA).


\chapter{Manual de usuario e instalación}
\label{cha:manual-usuario-instalacion}

\section{Introducción}
\label{sec:intro-manual-de-usuario}

En este apéndice se va a dividir en dos secciones diferenciadas donde se va a explicar como instalar y configurar todas las herramientas necesarias para la puesta a punto del proyecto. Y por otro lado, un manual de usuario donde se va a explicar como ejecutar cada uno de los algoritmos se han expuesto en el capítulo \ref{cha:desarrollo}.
Cabe destacar que, tal y como se indica en el sección \ref{sec:intro-pliego} del Apéndice \ref{cha:pliego-de-condiciones}, el equipo donde se instalará todo el software descrito en este apartado para programar los distintos algoritmos, se ha optado por utilizar un sistema operativo Ubuntu 18.04.4 LTS. Por tanto, todos los comandos de instalaciones y puesta en funcionamiento están orientados a un equipo que disponga de Linux.


\section{Guía de instalación}
\label{sec:sec-guia-instalacion}

\subsection{Instalación de Git}
\label{subsec:instalacion-git}

Git se trata de una herramienta de control de versiones distribuido de código que trabaja de una manera muy rápida y potente. Tiene un sistema para trabajar mediante ramas que pueden seguir una linea de progreso diferente a la principal, de tal manera que se pueden hacer pruebas del código o que distintas personas trabajen en ramas distintas y posteriormente, cuando haya una versión final, se pueda incluir en la rama principal. Para proyectos como el presente, es necesario de tener disponible historial completo de versiones para su correcto desarrollo.

Para poder instalar Git abra el terminal y ejecute los siguientes comandos:

\vspace{0.5cm}
\begin{lstlisting}[language=iPython,caption=Instalación de Git,captionpos=b,label={lst:install-git}]
# Actualizar paquetes de los repositorios
sudo apt-get update

# Instalacion de Git con todas sus depedencias
sudo apt-get install git-all
\end{lstlisting}

\subsection{Instalación de Anaconda}
\label{subsec:instalacion-anaconda}

Anaconda se trata de la \textit{suite} más compleja para la Ciencia de Datos en R y Python, lenguajes de programación que actualmente son líderes en Machine Learning, Inteligencia Artificial y Big Data. Esta \textit{suite} gratuita y multiplataforma dispone de las IDE's y librerías adecuadas para su manejo. Con ellos se evita tener que instalar manualmente un IDE y el lenguaje Python con sus correspondientes librerías que en ocasiones puede ser una operación tediosa y compleja.

\begin{enumerate}
    \item Accede al siguiente enlace \cite{inst-conda} disponible en la bibliografía y descargue la versión más reciente del instalador de Anaconda para Linux:
    
    \begin{figure}[ht]
    \centering
    \includegraphics[width=0.8\textwidth]{img/appendix/C/anaconda-installer.png}
    \caption{\label{fig:anaconda-download}Descarga del instalador de Anaconda}
    \end{figure}

    \item Abra el terminal y acceda a la carpeta \texttt{Downloads} donde ha descargado el instalador. Es recomendable verificar la integridad del instalador mediante la suma de comprobación SHA-256:
    
    \vspace{0.5cm}
    
    \begin{lstlisting}[language=iPython,caption=Verificación de la integridad de la instalación de Anaconda,captionpos=b,label={lst:verificar-sha256}]
    # Acceder a la carpeta Downloads del usuario
    cd /home/<username>/Downloads
    
    # Verificacion de la integridad del instalador
    sha256sum Anaconda3-2020.02-Linux-x86_64.sh
    \end{lstlisting}
    
    \item Ejecute la secuencias de comandos de Anaconda. Presione \texttt{yes} para aceptar los términos de la licencia. Pulse acontinuación \texttt{ENTER} cuando seleccione la ubicación de la instalación y finalmente presione \texttt{yes} para confirmar el \texttt{PATH} de Anaconda con tu \texttt{.bashrc}:
    
    \vspace{0.5cm}
    
    \begin{lstlisting}[language=iPython,caption=Ejecutar el instalador de Anaconda para Linux,captionpos=b,label={lst:install-conda}]
    # Ejecutar el instalador de Anaconda para Linux
    bash Anaconda3-2020.02-Linux-x86_64.sh
    \end{lstlisting}
    
    \item Cuando se complete la instalación cierre y abra el terminal de nuevo, o bien ejecute el siguiente comando:
    
    \vspace{0.5cm}
    
    \begin{lstlisting}[language=iPython,caption=Hacer efectivo los cambios en el fichero .bashrc,captionpos=b,label={lst:source-bashrc}]
    # Hacer efectivos los cambios realizados en .bashrc
    source /home/<username>/.bashrc
    \end{lstlisting}
\end{enumerate}

\subsection{Descarga del repositorio del proyecto}
\label{subsec:descarga-repo}

En esta subsección se indica como descargar el repositorio donde se ha programado los algoritmos de detección de objetos abandonados. Para ello se debe de seguir los comandos que se muestran a continuación:

\vspace{0.5cm}

\begin{lstlisting}[language=iPython,caption=Descarga repositorio,captionpos=b,label={lst:descarga-repo}]
# Acceder a la carpeta Documents del usuario
cd /home/<username>/Documents

# Clonar el repositorio de GitHub
git clone https://github.com/jmudy/yolov4-deepsort.git

# Acceder dentro de la carpeta clonada
cd yolov4-deepsort
\end{lstlisting}

\subsection{Crear entorno virtual con Anaconda}
\label{subsec:creacion-entorno}

Con la finalidad de evitar tediosa y larga instalación de CUDA y cuDNN y tener únicamente instaladas las librerías necesarias para que funcione el código se va a instalar un entorno virtual en Anaconda.

Desde la misma carpeta del repositorio de Github \ref{lst:descarga-repo} que se ha clonado hay dos ficheros con extensión .yml, donde viene indicadas la versión de Python que se va a necesitar, la versión de CUDA y cuDNN así como las librerías y dependencias necesarias. Para utilizar CUDA es necesario disponer de una GPU de NVIDIA. Es muy importante verificar la capacidad de cálculo de la tarjeta gráfica NVIDIA que se emplea. En la siguiente dirección \cite{cuda-gpus} se puede comprobar la capacidad de cálculo los distintos modelos:

La capacidad de cálculo mínima para poder utilizar CUDA con al librería tensorflow-gpu es de 3.5. En función de la GPU que dispongas, ejecute uno de los siguientes comandos para crear un entorno virtual de Anaconda con Python 3.7.0:

\vspace{0.5cm}

\begin{lstlisting}[language=iPython,caption=Creación entorno virtual en Anaconda,captionpos=b,label={lst:crear-env}]
# Si tu GPU tiene una capacidad de calculo < 3.5
conda env create -f conda-cpu.yml

# Si tu GPU tiene una capacidad de calculo >= a 3.5
conda env create -f conda-gpu.yml
\end{lstlisting}

Una vez instalado el entorno virtual se puede acceder a él ejecutando el siguiente comando:

\vspace{0.5cm}

\begin{lstlisting}[language=iPython,caption=Activar entorno virtual de Anaconda,captionpos=b,label={lst:activar-env}]
# Si instalaste el entorno con conda-cpu.yml
conda activate yolov4-cpu

# Si instalaste el entorno con conda-gpu.yml
conda activate yolov4-gpu
\end{lstlisting}

\subsection{Descargar las bases datos}
\label{subsec:descarga-datasets}

Los enlaces para descargar las bases de datos que se han utilizado para evaluar dos distintos algoritmos a lo largo del proyecto se encuentran a continuación:

\begin{itemize}
    \item ABODA dataset \url{https://github.com/kevinlin311tw/ABODA} \cite{aboda-dataset}
    \item AVENUE dataset \url{http://www.cse.cuhk.edu.hk/leojia/projects/detectabnormal/dataset.html} \cite{avenue-dataset}
    \item AVSSAB2007 dataset \url{http://www.eecs.qmul.ac.uk/~andrea/avss2007_d.html} \cite{AVSSAB2007-dataset}
    \item PETS2006 dataset \url{http://www.cvg.reading.ac.uk/PETS2006/data.html} \cite{pets2006-dataset}
    \item PETS2007 dataset \url{http://www.cvg.reading.ac.uk/PETS2007/data.html} \cite{pets2007-dataset}
    \item GBA dataset \url{https://bit.ly/3aVlGZL} \cite{gba-dataset}
\end{itemize}

\section{Manual de usuario}
\label{sec:manual-usuario}

En esta parte se especifica los comandos que se deben que ejecutar para poner en funcionamiento el algoritmo de detección de objetos, el algoritmo de seguimiento o \textit{tracking} de personas y objetos y por último el algoritmo el cual determinar si un objeto ha sido abandonado o perdido.
\pendiente{Queda pendiente este apartado que podrá ser completado cuando termine de redactar los apartados de desarrollo de algoritmo de detección de objetos abandonados}

\subsection{Ejecutar algoritmo de detección de objetos YOLOv4}

\vspace{0.5cm}
\begin{lstlisting}[language=iPython,caption=Funciones de object tracking ,captionpos=b,label={lst:object-tracking}]
def init_deepsort_params():
    # Definition of the parameters
    max_cosine_distance = 0.4
    nn_budget = None

    model_filename = 'model_data/mars-small128.pb'
    encoder = gdet.create_box_encoder(model_filename, batch_size=1)
    # calculate cosine distance metric
    metric = nn_matching.NearestNeighborDistanceMetric("cosine", max_cosine_distance, nn_budget)
    # initialize tracker
    tracker = Tracker(metric)
    return encoder, tracker
    
def load_obj_detector_cfg():
    config = ConfigProto()
    config.gpu_options.allow_growth = True
    session = InteractiveSession(config=config)
    STRIDES, ANCHORS, NUM_CLASS, XYSCALE = utils.load_config(FLAGS)
    input_size = FLAGS.size
    video_path = FLAGS.video
    return input_size, video_path

\end{lstlisting}

Probamos referenciar el código \ref{lst:install-git}, el código \ref{lst:descarga-repo} y el también el código \ref{lst:object-tracking}

Ejemplo por si hay que usar alguna variable = a un número:

\begin{lstlisting}[language=iPython]
<- #there shouldn't be quotation marks
"""
---------
sin2_theta  = np.sin(theta)**2
"""
import math
import numpy as np
from lib.analytical import csa

sin2_theta  = np.sin(theta)**2
+= -= *= /= + - * / ? < > & % == <=
# += -= *= /= + - * / ? < > & % == <=
def test(a=¢100¢, b=True)
    <= >= == ¢2¢ + ¢3j¢ * ¢7e-3¢
\end{lstlisting}

\end{appendices}

% Contraportada
\includepdf[pages={6}]{cover/portada.pdf}

\end{document}
