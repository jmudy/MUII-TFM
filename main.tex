
\documentclass[12pt]{article}

\usepackage[utf8]{inputenc}
\usepackage[english]{babel}
\usepackage{lipsum} % Para meter texto random

\title{Design, Implementation and evaluation of a strategy for detecting abandoned objects in video surveillance applications}
\author{Jesús Mudarra Luján}
\date{\today} % De esta manera la fecha siempre será la actual y no una en concreto

\begin{document}

\maketitle % Introducimos el título, autor y fecha
\tableofcontents % Para insertar el índice de contenidos


\begin{abstract}
    \lipsum[2]
\end{abstract}


\section{Introduction}
Ya hemos visto que se trata de un lenguaje perfecto para poder escribir texto con fórmulas matemáticas.

El primer parráfo después de una sección o subsección no se indenta, los párrafos posteriores sí.


\section{Section 2}
\lipsum[7-9]


\subsection{Subsection bla bla bla}
\lipsum[10-12]


\section{Section 3}
\lipsum[13-15]


\section{Section 4}
\lipsum[16-18]


\section{Section 5}
\lipsum[20-22]


\end{document}
