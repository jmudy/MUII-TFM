\section{Estudio teórico}

\subsection{Técnicas utilizadas}

\begin{frame}{Estudio teórico}{Técnicas utilizadas - Segmentación de objetos en primer plano}

\begin{itemize}
    \item blablabla
    \item blablabla
    \item blablabla
    \item blablabla
\end{itemize}


\begin{figure}[ht]
\centering
\includegraphics[width=0.4\textwidth]{Images/estudio-teorico/background-subtraction-example.jpg}
\caption{\label{fig:background-subtraction-example}Ejemplo sustracción del fondo}
\end{figure}

\end{frame}

%%%%%%%%%%%%%%%%%%%%%%%%%%%%%%%%%%%%%%%%%%%%%%%%%%%%%%%%%%%%%%%%%

\begin{frame}{Estudio teórico}{Técnicas utilizadas - Detección de objetos estacionarios}

\begin{figure}[ht]
\centering
\includegraphics[width=1\textwidth]{Images/estudio-teorico/metodos-sustraccion-fondo-deteccion-fondo-estacionario.png}
\caption{\label{fig:metodos-sustraccion-fondo-deteccion-fondo-estacionario}Clasificación de sustracción del fondo basados en métodos de detección de objetos estacionarios}
\end{figure}

\end{frame}

%%%%%%%%%%%%%%%%%%%%%%%%%%%%%%%%%%%%%%%%%%%%%%%%%%%%%%%%%%%%%%%%%

\begin{frame}{Estudio teórico}{Técnicas utilizadas - Reconocimiento del comportamiento}

\begin{itemize}
    \item blablabla
    \item blablabla
    \item blablabla
    \item blablabla
\end{itemize}

\begin{columns}

  \begin{column}{0.5\textwidth}
    \begin{figure}[ht]
    \centering
    \includegraphics[width=0.65\textwidth]{Images/estudio-teorico/Abnormal-behaviors-single-person.jpg}
    \caption{\label{fig:Abnormal-behaviors-single-person}Comportamientos anómalos en una persona}
    \end{figure}
  \end{column}
  
  \begin{column}{0.5\textwidth}  %%<--- here
    \begin{figure}[ht]
    \centering
    \includegraphics[width=0.5\textwidth]{Images/estudio-teorico/abnormal-behaviors-crowded-scene.jpg}
    \caption{\label{abnormal-behaviors-crowded-scene}Comportamientos anómalos en múltiples personas}
    \end{figure}    
  \end{column}
  
\end{columns}

\end{frame}

%%%%%%%%%%%%%%%%%%%%%%%%%%%%%%%%%%%%%%%%%%%%%%%%%%%%%%%%%%%%%%%%%

\begin{frame}{Estudio teórico}{Técnicas utilizadas - Detección de personas y objetos}

\begin{itemize}
    \item blablabla
    \item blablabla
    \item blablabla
    \item blablabla
\end{itemize}

\begin{figure}[ht]
\centering
\includegraphics[width=0.4\textwidth]{Images/estudio-teorico/people-detection-classification.png}
\caption{\label{fig:people-detection-classification}Clasificación en la detección de personas}
\end{figure}

\end{frame}

%%%%%%%%%%%%%%%%%%%%%%%%%%%%%%%%%%%%%%%%%%%%%%%%%%%%%%%%%%%%%%%%%

\subsection{Algoritmos de detección y seguimiento elegidos}

\begin{frame}{Estudio teórico}{Algoritmo de detección de objetos YOLOv4}

\justifying
En vista a las métricas se puede concluir que, de todos los detectores de objetos en tiempo real actuales, YOLOv4 es el mejor en términos de velocidad y precisión.

\begin{figure}[ht]
\centering
\includegraphics[width=0.5\textwidth]{Images/estudio-teorico/yolov4-vs-others.png}
\caption{\label{fig:yolov4-vs-others}YOLOv4 frente a otros detectores}
\end{figure}

\end{frame}

%%%%%%%%%%%%%%%%%%%%%%%%%%%%%%%%%%%%%%%%%%%%%%%%%%%%%%%%%%%%%%%%%

\begin{frame}{Estudio teórico}{Algoritmo de detección de objetos YOLOv4}

\begin{itemize}
    \justifying
    \item La detección primero divide la imagen en una cuadrícula de SxS.
    \item En cada una de las celdas se predice N posibles cuadros delimitadores y se calcula el nivel de certidumbre.
    \item Se eliminan las cajas que estén por debajo de un límite.
    \item A los cuadros restantes se les aplica un paso de ``non-max suppression''.
\end{itemize}

\begin{figure}[ht]
\centering
\includegraphics[width=0.49\textwidth]{Images/estudio-teorico/funcionamiento-yolo.jpg}
\caption{\label{fig:funcionamiento-yolo}Funcionamiento del algoritmo YOLO}
\end{figure}

\end{frame}

%%%%%%%%%%%%%%%%%%%%%%%%%%%%%%%%%%%%%%%%%%%%%%%%%%%%%%%%%%%%%%%%%

\begin{frame}{Estudio teórico}{Algoritmo de seguimiento de objetos Deep SORT}

\begin{itemize}
    \justifying
    \item Utiliza los pesos pre-entrenados de YOLOv4 para extraer los cuadros delimitadores de las detecciones.
    \item Los cuadros delimitadores sirven de entrada al algoritmo SORT para realizar el seguimiento. % Algoritmo SORT basado en el Filtro de Kalman
    \item Mediante una \textbf{DNN} entrenada el algoritmo asocia los \textit{tracks} activos y los que se han perdido.
\end{itemize}

\begin{figure}[ht]
\centering
\includegraphics[width=0.75\textwidth]{Images/estudio-teorico/yolo-deepsort-scheme.jpg}
\caption{\label{fig:yolo-deepsort-scheme}Esquema etapas detección y seguimiento con YOLOv4 y Deep SORT}
\end{figure}

\end{frame}
