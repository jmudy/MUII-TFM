\section{Introducción}

\begin{frame}{Introducción}

\begin{block}{Sistemas de videovigilancia automatizados}
\justifying
Han despertado un gran interés en los últimos años en la monitorización de lugares públicos y privados.
\vspace{0.2cm}
\end{block}

\begin{itemize}
    \justifying
    \item En los últimos tiempos la detección de objetos se ha investigado para detectar eventos de gran interés como \textbf{objetos abandonados} y \textbf{vehículos estacionados} ilegalmente.
    \item Los desarrollos de técnicas de detección de objetos abandonados se encuentran constantemente enfrentados contra diferentes \textbf{desafíos} durante su implementación.
    \item Los sistemas de detección de objetos abandonados actuales se centran principalmente en dos etapas principales: \textbf{detección estacionaria} y \textbf{clasificación de objetos}.
\end{itemize}

\end{frame}

%%%%%%%%%%%%%%%%%%%%%%%%%%%%%%%%%%%%%%%%%%%%%%%%%%%%%%%%%%%%%%%%%

\begin{frame}{Introducción}

\justifying
Los objetos abandonados se pueden determinar mediante dos reglas: el objeto aspirante se encuentra estático o desatendido.

\vspace{0.1cm}

\begin{itemize}
    \justifying
    \item El primer enfoque corresponde a una \textbf{regla espacial}, en la que un objeto se considera desatendido si el propietario del objeto se encuentra apartado del objeto.
    
\end{itemize}
    
\vspace{0.3cm}

\begin{figure}[ht]
\centering
\includegraphics[width=0.4\textwidth]{Images/introduccion/pets2006-3m.jpeg}
\caption{\label{fig:pets2006-3m}Persona sobrepasando la zona de alarma (marcado en amarillo)}
\end{figure}
    
\end{frame}

%%%%%%%%%%%%%%%%%%%%%%%%%%%%%%%%%%%%%%%%%%%%%%%%%%%%%%%%%%%%%%%%%

\begin{frame}{Introducción}

\begin{itemize}
    \justifying
    \item El segundo enfoque define una \textbf{regla temporal} en la que un objeto se considera estacionario si se encuentra inmóvil durante un cierto período de tiempo, dependiendo de la aplicación, siendo típicamente 30 o 60 segundos.
    
\end{itemize}
    
\vspace{0.3cm}

\begin{figure}[ht]
\centering
\includegraphics[width=0.85\textwidth]{Images/introduccion/canonical-framework-aod.png}
\caption{\label{fig:canonical-framework-aod}Marco de referencia en detección de objetos abandonados}
\end{figure}

\end{frame}

%%%%%%%%%%%%%%%%%%%%%%%%%%%%%%%%%%%%%%%%%%%%%%%%%%%%%%%%%%%%%%%%%

\begin{frame}{Introducción}

\justifying
En los últimos años se ha mostrado un \textbf{gran progreso} en la detección de personas debido a la aparición de métodos de aprendizaje profundo o \textit{Deep Learning}.

\begin{figure}[ht]
\centering
\includegraphics[width=0.65\textwidth]{Images/introduccion/Block-diagram-Stationary-detection.png}
\caption{\label{fig:diagram-stationary-detection}Diagrama de bloques del módulo de generación de candidatos}
\end{figure}
    
\end{frame}

%%%%%%%%%%%%%%%%%%%%%%%%%%%%%%%%%%%%%%%%%%%%%%%%%%%%%%%%%%%%%%%%%

\subsection{Objetivos}

\begin{frame}{Introducción}{Objetivos}

\begin{exampleblock}{Una meta}
\justifying
El objetivo que se persigue es el desarrollo de una estrategia de detección de objetos \textbf{abandonados} mediante el uso de \textbf{CNN} en aplicaciones de videovigilancia en espacios como aeropuertos, estaciones de metro, edificios o cualquier tipo de infraestructura que disponga de una o varias cámaras de videovigilancia.
\end{exampleblock}

\vspace{0.3cm}
Para alcanzar dicho objetivo, se han planteado los siguientes puntos:
\vspace{0.1cm}

\begin{itemize}
    \justifying
    \item Revisión del Estado de Arte.
    \item Evaluación de algoritmos de detección de objetos más relevantes.
    \item Evaluación de algoritmos de seguimiento o de objetos más relevantes.
    \item Implementación y evaluación de un algoritmo de detección de objetos abandonados.
\end{itemize}
  
\end{frame}