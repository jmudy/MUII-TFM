
\chapter{Presupuesto}
\label{cha:presupuesto}

\section{Equipo de trabajo}

Para la realización del proyecto se va a necesitar un Ingeniero de Telecomunicaciones.

\section{Timing}

Las fases necesarias para la realización del desarrollo son las siguientes.

\begin{itemize}
  \item Configuración del hardware (0,5 meses)
  
  \begin{itemize}
    \item Instalación de sistema operativo, librerías, IDE. (0,25 meses)
    \item Configuración y puesta en marcha. (0,25 meses)
  \end{itemize}
  
\item Diseño e implementación de los módulos software necesarios (6 meses):
  
  \begin{itemize}
    
  \item Diseño e implementación de librerías de soporte para la
    reproducción y adquisición de audio multicanal temporizado (tiempos
    fijos) (2 mes).
    
  \item Diseño e implementación de librerías de soporte para la
    adquisición, reproducción y procesamiento de audio multicanal
    continuo (tiempo indefinido) (4 mes).
    
  \end{itemize}
  

\item Evaluación del sistema de procesamiento de audio multicanal (1,5 meses):
  
  \begin{itemize}
    
  \item Adaptación del algoritmo SRP. (0,5 meses)
  
  \item Evaluación del sistema de adquisición sobre la librería de tiempo real (1 mes)
    
    \end{itemize}
  
\item Documentación

\end{itemize}

Por supuesto, las fases de diseño, desarrollo, pruebas y documentación
son cíclicas y abarcan todo el periodo de la vida del proyecto.


\section{Presupuesto total}

Por todo lo anterior, supone una duración de 8 meses naturales con un coste de 12.000 \euro (sin IVA).
